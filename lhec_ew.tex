
%-----------------------------------------------------------------------
%   Introduction
%-----------------------------------------------------------------------
\section{Introduction}
Since the discovery of the Standard Model (SM) Higgs boson at the CERN
LHC experiments and subsequent measurement of its
parameters, all fundamental parameters of the SM have been measured
directly and with remarkable precision.
To further establish the validity of the theory of electroweak
interactions, validate the mechanism of electroweak symmetry breaking
and the mechanism of the generation of particle masses, further
high-precision electroweak measurements have to be performed.
Such high-precision measurements are also often considered as a portal
to new physics, since non-SM contributions, as for instance
loop-insertions, may yield significant deviations for some precisely
measurable and calculable observables.
The greatly enlarged kinematic reach to higher scales in comparison to
HERA and the large targeted luminosity will allow for the first time
high-precision electroweak measurements in $ep$.

The LHeC experimental conditions offer the opportunity for unique measurements of electroweak
parameters, which are often complementary to other experiments, such
as proton-proton or electron-positron collider experiments or low
energy neutrino or muon scattering experiments.
Among many other quantities, unique measurements of the weak couplings
of the light quarks, $u$ and $d$, can be performed due to the
important contributions of valence quarks in the initial state, as well as scale
dependent measurements of weak interactions, since deep-inelastic $ep$
scattering is mediated through space-like momentum transfer
($t$-channel exchange).

%
In this article we study the sensitivity of inclusive NC and CC cross
section at LHeC to electroweak parameters.



%-----------------------------------------------------------------------
%   Theory
%-----------------------------------------------------------------------
\clearpage
\section{Electroweak effects in inclusive NC and CC DIS cross sections}
%
Inclusive NC DIS cross sections are expressed in terms of
generalised structure functions $\tilde{F}_2^\pm$, $x\tilde{F}_3^\pm$
and $\tilde{F}_{\rm L}^\pm$ at EW leading order (LO)
as
\begin{equation}
  \frac{d^2\sigma^{\rm NC}(e^\pm p)}{dxd\Qsq} = \frac{2\pi\alpha^2}{xQ^4}\left[Y_+\tilde{F}_2^\pm(x,\Qsq) \mp Y_{-}  x\tilde{F}_3^\pm(x,\Qsq) - y^2 \tilde{F}_{\rm L}^\pm(x,\Qsq)\right]~,
  \label{eq:cs}
\end{equation}
where $\alpha$ denotes the fine structure constant and $x$ the
Bjorken scaling variable.
The terms $Y_\pm = 1\pm(1-y)^2$ contain the helicity dependence of
the process, where $y$ denotes the inelasticity of the process.
The  generalised structure functions are then separated into contributions
from pure $\gamma$- and $Z$-exchange and their interference~\cite{Klein:1983vs}:
\begin{align}
  \tilde{F}_2^\pm
  &= F_2
  -(\ve\pm P_e\gae)\varkappa_ZF_2^{\gamma Z}
  +\left[(\ve\ve+\gae\gae)\pm2P_e\ve\gae\right]\varkappa_Z^2F_2^Z~,
  \\
  \tilde{F}_3^\pm
  &=~~
  -(\gae\pm P_e\ve)\varkappa_ZF_3^{\gamma Z}
  +\left[2\ve\gae\pm P_e(\ve\ve+\gae\gae)\right]\varkappa_Z^2F_3^Z~.
  \label{eq:strfun}
\end{align}
Similar expressions hold for $\tilde{F}_L$.
%
In the naive quark-parton model, which corresponds to the LO QCD
approximation, the structure functions are calculated as
\begin{align}
  \left[F_2,F_2^{\gamma Z},F_2^Z\right]
  &= x\sum_q\left[Q_q^2,2Q_q\vq,\vq\vq+\aq\aq \right]\{q+\bar{q}\}~,
  \label{eq:last1}
  \\
  x\left[F_3^{\gamma Z},F_3^Z\right]
  &= x\sum_q\left[2Q_q\aq,2\vq\aq\right]\{q-\bar{q}\}~,
  \label{eq:last2}
\end{align}
and it is easily recognized that those are closely related
to the quark and anti-quark momentum distributions, $xq$ and $x\bar{q}$.
In eqs.~\eqref{eq:strfun} and~\eqref{eq:gV-LO}, the variables $g_V^{e/q}$
and $g_A^{e/q}$ stand for the vector and axial-vector couplings of the
lepton or quarks to the $Z$ boson, and the coefficient $\varkappa_Z$
accounts for the $Z$-boson propagator and the normalisation of the  weak contributions.
%
Both parameters are given by electroweak theory.
The (effective) coupling parameters
depend on the electric charge, $Q_{q/e}$, in units of the positron charge,
and on the third component of the weak-isospin of the fermion, $I^3_{{\rm L},q/e}$.
Using $\sw=1-\tfrac{M_W^2}{M_Z^2}$, they are given by
\begin{align}
  g_A^{f} &= \sqrt{\rho_{\text{NC}, f}} I^3_{{\rm L},f}
  \label{eq:gA-LO} \,, \\
  g_V^{f} &= \sqrt{\rho_{\text{NC}, f}} \left(I^3_{{\rm L},f} - 2
  Q_{f} \kappa_{\text{NC}, f}\ \sw \right)
  \label{eq:gV-LO} \,.
\end{align}
The form factors $\rho_{\text{NC}, f}$ and $\kappa_{\text{NC}, f}$
represent the universal higher-order corrections, and due to their
\Qsq\ dependence the couplings become `effective'.
%The latter has been measured with highest precision
%in muon-decay experiments~\cite{Tishchenko:2012ie}.
%
The coefficient $\varkappa_Z$ is calculated as
\begin{equation}
  \varkappa_Z(\Qsq)
  = \frac{\Qsq}{\Qsq+M^2_Z}
  \frac{1}{4\sw \cos^2\theta_W}
  = \frac{\Qsq}{\Qsq+M^2_Z}
  \frac{\gf M_Z^2}{2\sqrt{2}\pi\alpha}~,
\end{equation}
%i.e.\ taking into account either the weak mixing angle, $\sw=1-\mW^2 /
i.e.\ taking into account $\sw$, or alternatively using the Fermi coupling constant $\gf$.

In order to perform predictions in LO electroweak theory only two independent
parameters are needed in addition to $\alpha$.
At higher orders, loop corrections involve a non-negligible dependence
on further parameters, where the most important ones are $M_t$
and $M_H$ and hadronic contributions.


In the LO approximation, the CC DIS cross section is written as
\begin{equation}
  \frac{d^2\sigma^{\rm CC}(e^\pm p)}{dxd\Qsq}
  = \left(1 \pm P_e\right)
  \frac{\gf^2}{4\pi x}
  \left[\frac{m_W^2}{m_W^2+\Qsq}\right]^2
  \left(Y_+ W_2^\pm(x,\Qsq) \mp Y_{-} xW_3^\pm(x,\Qsq)
  - y^2 W_{\rm L}^\pm(x,\Qsq)\right)~.
  \label{eq:cc-cs}
\end{equation}
In the simplified quark-parton model the structure
functions $W_2^\pm$ and $xW_3^\pm$ are obtained from the parton
distribution functions, while $W_{\rm L}^\pm = 0$:
An incoming electron can scatter only with
positively charged quarks,
\begin{equation}
  W_2^- =
  x \left( U + \overline{D} \right)
  \, ,
  \quad xW_3^- =
  x \left( U - \overline{D} \right)
  \, ,
  \label{eq:w23el-LO}
\end{equation}
while positrons scatter only with negatively charged quarks,
\begin{equation}
  W_2^+ =
  x \left( \overline{U} + D \right)
  \, {~~\text{and}~~}
  \quad xW_3^+ =
  x \left( D - \overline{U} \right)
  \, ,
  \label{eq:w23po-LO}
\end{equation}
using the parton combinations
$U = u+c$, $\overline{U} = \bar{u} + \bar{c}$, $D = d+s$
and $\overline{D} = \bar{d} + \bar{s}$.
{\color{green} Introduce \Qsq dependent form factors.}

Due to renormalisation of the corrections and the choice of
input parameters the calculations become scheme dependent.
In this study we adopt the on-shell scheme using \mz\ and \mw\ as input
parameters to the calculations.


% -------------------------------------------------------------------
%
% -------------------------------------------------------------------
\clearpage
\section{The inclusive DIS cross section and LHeC}
In order to illustrate the contribution of electroweak effects to the
inclusive DIS cross sections, 
the single-differential inclusive NC and CC DIS cross sections
for polarised $e^-p$ scattering as function of \Qsq for LHeC electron
and proton beam energies are displayed in
figure~\ref{fig:dSigma}.
Electron beam energies of $E_e=50\,\GeV$ and 60\,\GeV are
displayed. 
The LHeC predictions are compared to measurements at HERA
($E_e=27.6\,\GeV$ and $E_p=920\,\GeV$, with $P_e=0$) and their
predictions. 
%
\begin{figure}[htb]
  \centering
  \includegraphics[width=0.52\textwidth]{plot_GraphsDsigmaDQ2_LHeC}
  \caption{{Single differential inclusive DIS cross sections for
      polarised $e^-p$ NC 
      and CC DIS at the LHeC for two different electron beam energies
      ($E_e=50\,\GeV$ and 60\,\GeV). Cross sections for longitudinal
      electron beam
      polarisations of $P_e=-0.8$ and $+0.8$ are displayed.
      For comparison also measurements at center-of-mass energies
      of $\sqrt{s}=920$ by H1 at HERA for unpolarised ($P=0\,\%$)
      electron beams are displayed.
  }}
  \label{fig:dSigma}
\end{figure}
%
At lower values of \Qsq, the NC cross sections are dominated by the
photon-exchange contribution to the cross sections represented by and
the structure function $F_2$ (c.f.~Eq.\,\eqref{eq:strfun}), and the CC
cross section are suppressed with respect to the NC ones and become
almost independent on \Qsq\ due to the mass of the $W$ boson, since
$\Qsq\ll\mW^2$ and thus the propagator term becomes
$\tfrac{\mW^2}{\mW^2+\Qsq} \simeq 1$.
At \Qsq\ values around the electroweak scale, $\Qsq\approx\mZ^2$,
weak contributions to the NC cross sections become important.
For instance, this can be observed as the NC cross section becomes
dependent on the longitudinal beam polarisation, $P_e$, and the cross
sections for positive and negative polarisations start to differ
significantly.
For CC, the impact of the longitudinal beam polarisation is of course
much more prominent, since CC is mediated exclusively by weak
interactions and the cross section scales linearly with the
size of $P_e$ (c.f.~Eq.\,\eqref{eq:cc-cs}).
%Also for NC cross sections the 
%polarisation effects are significant in kinematic regions, where
%$\gamma Z$ and pure $Z$  exchange are important.
With a reduced electron beam energy of $E_e=50\,\GeV$, the
center-of-mass energy reduces to $1.18\,\TeV$ and
in the exemplary range from $10\,000<\Qsq<100\,000\,\GeVsq$
the reduced NC or CC cross sections are lower by about 10 to 15\,\%.
At even higher \Qsq\ values the difference further increases.
%and also the
%sensitivity to electroweak parameters in NC exchange reduces
%accordingly.
%

\clearpage
{\color{green}Two alternative versions of this plot are below}
\begin{figure}[h!b]
  \centering
  \includegraphics[width=0.38\textwidth]{plot_GraphsDsigmaDQ2_LHeC_opt}
  \hskip0.05\textwidth
  \includegraphics[width=0.38\textwidth]{plot_GraphsDsigmaDQ2_FCC_opt}
  \caption{
    Left: Inclusive NC and CC DIS cross sections as a function of \Qsq
    for two different polarisation states $P=\pm0.8$ and for two different electron
    beam energies, $E_e=50$ and $60\,\GeV$.
    For comparison, also the HERA measurements are displayed for $P=0$.
    Right: Inclusive NC and CC DIS cross sections as a function of
    \Qsq for different proton beam energies, $E_p=7$, 20 and
    50\,TeV. The latter two are possible at the future FCC.
    {\color{red} Appendix/additional figure?}
  }
  \label{fig:dSigmaopt}
\end{figure}


{\color{green}A potential additional plot could be about the
  polarisation asymmetry, but this maybe opens a long discussion...}
\begin{figure}[h!b]
\begin{center}
   \includegraphics[width=0.40\textwidth]{{figures/fcc_asymmetry}.pdf}
   \hskip1cm
   \includegraphics[width=0.40\textwidth]{{figures/fcc_CCtot}.pdf}
\end{center}
\caption{
  Left: Neutral-current polarisation asymmetry as a function of
  \Qsq\ integrated over $x$ for FCC-eh simulated data. The polarisation asymmetry
  is displayed for pure photon exchange, which is zero
  by definition, % due to the absence of parity violating terms.
  for calculations including only the interference terms $\gamma^*Z$,
  and for predictions including also purely weak effects, $ZZ$.
  The full circles illustrate simulated data points, which have
  uncertainties invisible at the chose scales.
  Right: Charged-current cross sections measured with different lepton
  polarisation states and for electron (red) and positron
  (blue) beams. The full circles illustrate FCC-eh simulated data,
  whereas the open circles show H1 measurements. The data are scaled
  by the the center-of-mass energies of the respective collider.
  The error bars are smaller than the markers.
}
\label{fig:crosssections}
\end{figure}



% -------------------------------------------------------------------
%
% -------------------------------------------------------------------
\clearpage
\section{LHeC pseudo-data}
\label{sec:data}
In this section, the details of the LHeC pseudo-data\,\footnote{In the
following, the pseudo-data is simply denoted as `data' in order to
facilitate reading.} as used for the
upcoming extraction of electroweak parameters are discussed.

In this analysis a recent simulation of inclusive NC and CC DIS cross
section data are exploited.
The data sets include electron and positron data, different lepton
beam polarisation states, and different proton beam energies.
Since the actual layout of the LHeC energy-recovery linac for the lepton
beam is not yet decided, we will study scenarios for two lepton beam
energies of either $E_e=50\,\GeV$ or 60\,\GeV in the following,
whereas the nominal LHC proton beam energy of $E_p=7000\,\GeV$ is considered.
For each beam setup NC and CC DIS data are simulated and
a summary of the data sets is given in Table~\ref{tab:datasets}.
\begin{table}[bht]
  \centering
  \small
  \begin{tabular}{lccccccc}
    \toprule
    Data set & Processes & $E_p$ [TeV] & $Q_e$ & $P_e$ &
    $\mathcal{L}$ [fb$^{-1}$] & \Qsq\ range [\GeVsq] & No. of data points (NC,CC)\\
    \midrule
    D1 & NC,CC &  7  & $-1$ &  $-0.8$ & 1000 & 5 -- $10^6$ & 100 \\ % D4
    D2 & NC,CC &  7  & $-1$ &  $+0.8$ & 10   & 5 -- 1\,000\,000 & 100 \\ % D8
    D3 & NC,CC &  7  & $+1$ &  $ 0  $ & 10   & 5 -- 1\,000\,000 & 100 \\ % D7
    D4 & NC,CC &  1  & $-1$ &  $ 0  $ & 1    & 5 -- 1\,000\,000 & 100 \\ % D5
    \bottomrule
 \end{tabular}
  \caption{Summary of simulated data sets used.
    Each data set is simulated for the two studied
    electron beam energies of $E_e=50\,\GeV$ and 60\,\GeV.
    {\color{red}last twp columns!}
    {\color{blue}to be checked, if Ee=50 and 60 used these settings.}
  }
  \label{tab:datasets}
\end{table}
The vast majority of the data will be collected with an electron
($Q_e=-1$) beam and with a beam longitudinal polarisation
of $P_e=-0.8$, reaching an integrated luminosity of about
$\mathcal{L}\simeq1000\,\textrm{fb}^{-1}$.
This will allow measurements of NC and CC DIS cross sections up to values of
$\Qsq\simeq1\,000\,000\,\GeVsq$.
A considerably smaller data sample will be collected with a positive
electron beam polarisation of $P_e=+0.8$. For this sample, an
integrated luminosity of 10\,fb$^{-1}$ is assumed.
Another data sample will be collected with a positron beam, where
an unpolarised beam is assumed.
The technical limitations for the generation of positrons put
constraints on the achievable beam current and thus on the instaneous
luminosity. Therefore, an
integrated luminosity of (only) 10\,fb$^{-1}$ is assumed for this sample.
Nonetheless, this will allow measurements with positrons up to
\Qsq\ values of 500\,000\,\GeVsq.
Yet another data sample will be collected with a reduce proton beam
energy, which will mainly be used for measurements of $F_L$ and access
higher values of $x$.
For this sample an integrated luminosity of 1\,fb$^{-1}$ is assumed.
All data is restricted to $\Qsq\geq5\,\GeVsq$ in order to avoid
regions, where higher order QCD effects are important, while the
low-\Qsq region has anyhow only very low sensitivity to EW parameters.


The simulated data points accout for the acceptance of the LHeC
detector, the kinematic reconstruction, and trigger restrictions.
The latter, for instance, restricts CC DIS measurements to
$\Qsq\gtrsim100\,\GeVsq$.
The resulting kinematic plane of the simulated data points is displayed, for
instance, in Ref.~\cite{AbdulKhalek:2019mps} (where, however, slightly
different assumptions on the data uncertainties have been used).

The data include a full set of systematic uncertainties and
the individual sources are summarised in Table~\ref{tab:uncert}.
\begin{table}[bht]
  \centering
  \small
  \begin{tabular}{lc}
    \toprule
    Source of uncertainty & Uncertainty \\
    \midrule
    Scattered electron energy scale $\Delta E_e' /E_e'$ & 0.1 \% \\
    Scattered electron polar angle  & 0.1\,mrad \\
    Hadronic energy scale $\Delta E_h /E_h$ & 0.5\,\% \\
    %calorimeter noise (only $y < 0.01$) & 1-3\,\% \\
    Radiative corrections & 0.3\,\% \\
    Photoproduction background ($y > 0.5$) & 1\,\% \\
    Uncorrelated uncertainty & 0.5\,\% \\ % Global efficiency error
    Luminosity uncertainty & 1.0\,\% \\
    \bottomrule
 \end{tabular}
\caption{
  Assumptions used in the simulation of the NC cross sections
  on the size of uncertainties from various sources. The top three are
  uncertainties on the calibrations which are transported to
  provide correlated systematic cross section errors.
  The lower three values are uncertainties of the cross section
  caused by various sources.
}
\label{tab:uncert}
\end{table}
For the bulk of the phase space, the `electron' reconstruction method
is employed and important uncertainties originate from the scattered
electron energy scale and its polar angle measurement, where
uncertainties of $\Delta E_e' /E_e'=0.1\,\%$ and
$\Delta\theta^\prime_e=0.1$\,mrad are assumed, respectively.
However, at lower values of $y$ the so-called `mixed' reconstruction
method~\cite{Blumlein:1990dj} is employed, which makes use of the
measurement of the hadronic final state for $y$, and $x=sy/\Qsq$. %$y_h=\tfrac{\Sigma}{2E_e}$.
For the measurement of the hadronic final state,
an uncertainty on the hadronic energy scale $\Delta E_h/E_h=0.5\,\%$
is imposed.
Furthermore, uncertainties on the QED radiative corrections of
0.3\,\%, and uncertainty on the background from photoproduction events
of $1.0\,\%$ at the high-$y$ region is assumed.
The statistical uncertainty is take to be at least 0.1\,\%.
A global normalisation uncertainty, including the luminosity
uncertainty, of 1\,\% is taken. 

A number of further potential uncertainty sources ({\color{red} e.g. ...}) is
summarised in an uncorrelated uncertainty component of 0.5\,\%.

These data samples have been simulated mainly for the purpose of performing
reliable studies for PDF determinations, which are mainly sensitive to
the lower \Qsq region.
In contrast, the present study on electroweak parameters is mainly
sensitive to the high \Qsq region.
For technical reasons, and for simplicity, an \emph{ad-hoc} $x$-\Qsq
grid for the simulated data points was chosen.
Though, the upcoming real data may allow a much finer binning, in
particular at medium $x$ values and higher \Qsq.
With the given treatement of fully correlated, or fully uncorrelated
uncertainty sources, a finer binning can to a very good approximation
be simulated by changing the size of the uncorrelated uncertainty.
In the following, an alternative scenarios is considered, where we assume
a two times finer bin-grid in $x$, as well as in \Qsq, which then is
simply emulated by using an uncorrelated uncertainty of $0.25\,\%$.
{\color{green} We keep a minimum statistical uncertainty on every data
  point of 0.1\,\%.}

%\footnote{Due to performance
%  reasons, the pseudo-data are generated for a rather coarse bin grid. With a
%  binning, which is closely related to the resolution of the LHeC
%  detector, a much finer binning in $x$ and \Qsq are feasible. Already
%  such a change would alter the uncertainties of the fit parameters.
%  However, such an effect can be reflected by a
%  changed uncorrelated uncertainty, and a value of 0.25\,\% appears
%  like an optimistic, but achievable,  alternative scenario. },

In summary, we study four sets of data samples, as summarised in Table\,\ref{tab:scenarios}.
\begin{table}[hbt]
  \centering
  \small
  \begin{tabular}{lccc}
    \toprule
    Number & $E_e$ & $\delta_\textrm{unc}$ & Bin grid \\
    \midrule
    1  &   50\,\GeV  &  0.5\,\%  &  nominal \\
    2  &   50\,\GeV  &  0.25\,\% &  4$\times$ nominal  \\
    3  &   60\,\GeV  &  0.5\,\%  &  nominal \\
    4  &   60\,\GeV  &  0.25\,\% &  4$\times$ nominal \\
    \bottomrule
 \end{tabular}
\caption{
  LHeC scenarios.
}
\label{tab:scenarios}
\end{table}



~\cite{Britzger:2017fuc}




%-----------------------------------------------------------------------
%   Methodology
%-----------------------------------------------------------------------
\clearpage
\section{Methodology of a combined EW and QCD fit}
The expected uncertainties of the electroweak parameters are
determined in a combined fit of the electroweak parameter(s) together
with the PDFs to the inclusive NC/CC DIS data (denoted in the following as `PDF+EW' fit), where their full set
of statistical and systematic uncertainties are considered.
In this fit, also parameters of the PDFs are determined, since at the
time of the LHeC, PDFs will predominantly be determined from these
inclusive DIS data under consideration.
In order to account for the correlation between uncertainties of the
PDFs, which mainly represenent the propagated uncertainties of the LHeC inclusive
DIS data, and the EW parameter, the PDFs have to be determined
simultaneously in this fit.

In the calculation, NNLO pQCD calculations with the zero-mass variable
flavor number scheme is used. 
For the PDF evolution or the structure function calculations, no QED
or EW contributions are included, since these are expected to be negligible.
In particular, these contributions will not add additional sensitivity
to the EW parameters of interest, and as such will not change the
expected uncertainties in the present study.
The $x$-dependent PDFs are parameterised at a scale of
$\mu_0=1.3784\,\GeV$, i.e.\ below the charm mass.
Five orthogonal PDF linear combinations are chosen to be parameterised
at $\mu_0$: the $u$ and
$d$-valence quark distributions ($xu$,$xv$), the $u$-type and $d$-type
anti-quark distributions ($x\bar{U}$,$x\bar{D}$), and the gluon
distribution ($xg$).
The chosen parameterisation follows closely previous LHeC PDF studies~\cite{todo},
which are closely related to HERAPDF-style PDFs~\cite{todo}.
The following functional form is chosen
\begin{equation}
  xf = f_A x^{f_B} (1-x)^{f_C} (1+f_Dx+f_Ex^2) -
      f_{A^\prime}x^{f_{B^\prime}}(1-x)^{0.25}\,,
\end{equation}
where $f$ denotes any of the five linear combinations.
The second summand is considered only for the gluon distribution\footnote{
Although, the second term is commonly considered to be of importance for PDF
determinations as it introduces
additional freedom at lower values of $x$ and since the
LHeC probes $x$ regions as low as $5\cdot10^{-6}$, it is found to has
no significant impact on the resulting uncertainties of the
electroweak parameters.}.
The normalisation parameters are determined through the quark number
sum-rule ($u_A$, $d_A$) or the momentum sum-rules ($A_g$), or are calculated
as $\bar{U}_A=\bar{D}_A(1-0.4)$.
Furthermore, we use $\bar{D}_B=\bar{U}_B$.
Finally, altogether 13 parameters  of the PDFs are determined in each
of the fits
($g_B$, $g_C$, $g_{A^\prime}$, $g_{B^\prime}$, $u_B$, $u_C$, $u_E$,
$d_B$, $d_C$, $\bar{U}_C$, $\bar{D}_A$, $\bar{D}_B$, $\bar{D}_C$), while
all other parameters are set to zero.
The actual values of the PDF parameters are not of noteworthy
relevance here, and are set to values, as they can be obtained from a
fit to HERA data~\cite{H1dataTodo}. 

EW effects are included into the calculation by considering the full
set of 1-loop electroweak corrections~\cite{todo}.
{\color{red} more to add here...}

The $\chi^2$ quantity which is input to the minimisation and
error propagation is based on normal-distributed relative
uncertainties,
\begin{equation}
  \chi^2 =\sum_{ij} \log{\frac{\varsigma_i}{\sigma_i}}V^{-1}_{ij}\log{\frac{\varsigma_j}{\sigma_j}}
\end{equation}
where the sum runs over all data points $\varsigma_i$ and their
corresponding predictions, $\sigma_i$.
The covariance matrix $V$ represents the relative uncertainties of the
data points.
The Minuit library is employed and the resulting uncertainties of the fit
parameters are calculated using the HESSE or MINOS algorithm~\cite{Minuit}.
For our study, we set the data values equivalent to the predictions,
i.e.\ our data represent an \emph{Asimov data set}~\cite{Cowan:2010js}.
It it is noteworthy, that with the chosen $\chi^2$ definition, the actual
size of the cross section at a given point does not enter the
calculation of the uncertainties, but only the relative size of the
uncertainties are of relevance.






%-----------------------------------------------------------------------
%   Results: mass parameters
%-----------------------------------------------------------------------
\clearpage
\section{Mass determinations of the weak bosons}
\label{sec:mass}
In this section, first, a determination of the $W$-boson mass, \mW,
then the $Z$-boson mass, \mZ, then a combined determination of \mW\ and
\mZ, and finally a possible determination of the Higgs-boson mass,
\mH, is discussed. 
The expected uncertainties for a determination of the weak boson
masses are determined in the PDF+EW fit, where one of the masses is
determined together with the PDFs.
The other mass parameters are taken as external input in this
fit.


For a determination of the $W$-boson mass, expected uncertainties of
\begin{alignat}{2}
  \Delta\mW(\text{LHeC-60})&=\pm5_{({\rm exp})}\pm8_{({\rm PDF})}\,\MeV  =\,&& 10_{\text{(tot)}}\,\MeV{\rm ~~and~~} \\
  \Delta\mW(\text{LHeC-50})&=\pm8_{({\rm exp})}\pm9_{({\rm PDF})}\,\MeV  =\,&& 12_{\text{(tot)}}\,\MeV
\nonumber
\end{alignat}
are found for the LHeC with $E_e=60\,\GeV$ or 50\,\GeV, respectively.
%
With the assumption of an uncorrelated uncertainty of $0.25\,\%$ (c.f.\
section\,\ref{sec:data}) uncertainties of
\begin{alignat}{2}
  \Delta\mW(\text{LHeC-60})&=\pm3_{({\rm exp})}\pm5_{({\rm PDF})}\,\MeV =\,&& 6_{\text{(tot)}}\,\MeV{\rm ~~and~~} \\
  \Delta\mW(\text{LHeC-50})&=\pm6_{({\rm exp})}\pm6_{({\rm PDF})}\,\MeV =\,&& 8_{\text{(tot)}}\,\MeV
\nonumber
\end{alignat}
for LHeC-60 and LHeC-50 are obtained, respectively.
The expected total uncertainties are displayed in
figure~\ref{fig:mW} (left) and compared to the values obtained by
LEP2~\cite{Schael:2013ita}, Tevatron~\cite{Group:2012gb},
ATLAS~\cite{Aaboud:2017svj} and the PDG value~\cite{PDG}.
The full dependence of the expected total experimental uncertainty
$\Delta\mW$ on the size of the uncorrelated uncertainty component
%and the normalisation uncertainty
is displayed in figure~\ref{fig:mW} (right).
\begin{figure}[thbp]
    \centering
    \includegraphics[width=0.42\textwidth]{alphas_summary_W-bosonMass}
    \hskip0.05\textwidth
    \includegraphics[width=0.42\textwidth]{mW_uncorr}
    \caption{
      Left: Measurements of the $W$-boson mass  at fixed top-quark
      and $Z$-boson masses at LHeC for different LHeC
      scenarios in comparison with todays
      measurements~\cite{Group:2012gb,Schael:2013ita,Aaboud:2017svj}
      and the  world average value (PDG19)~\cite{Tanabashi:2019}.
      For LHeC, prospects for $E_e=60\,\GeV$ and 50\,\GeV are
      displayed, as well as results for the two scenarios with 0.5\,\%
      or 0.25\,\% uncorrelated uncertainty (see text).
      Right: Comparison of the precision for $M_W$ for different assumptions of the uncorrelated
      uncertainty of the pseudo-data.
      The uncertainty of the world average value is displayed as horizontal line.
      The nominal (and alternative) size of the uncorrelated uncertainty of the
      inclusive NC/CC DIS pseudo-data is indicated by the vertical
      line (see text),
    }
    \label{fig:mW}
\end{figure}

An artificial breakdown into experimental and PDF uncertainties is
obtained by repeating the fit with PDF parameters kept fixed, which
yields the \emph{exp} uncertainty, while the PDF uncertainty is then
calculated as the quadratic difference from the total uncertainty.
The size of the uncertainty component associated to the PDFs is found
to be  of similar size as the \emph{exp} uncertainty.
%
It is found, that the LHeC measurement will yield the smallest
experimental uncertainties in a single experiment\,\footnote{In
  figure~\ref{fig:mW}, the values from LEP2 and Tevatron represent
  combined results taking into account measurements from independent
  experiments and thus benefit from a reduction of the systematic
  uncertainties in the combination procedure. Similarly, the
  PDG world average value.}.
The LHeC measurement will further be even superior than the current
world average. 
As such, a detailed assessement of associated theoretical
uncertainties will be needed, where the largest source is due to the
top-quark mass due its contribution to \dr. An uncertainty of the
top-quark mass of $0.5\,\GeV$ will yield an additional uncertainty of
$\Delta\mW=2.5\,\MeV$. 


{\color{blue}Conclusion?)
  The measurement of \mW\ in DIS constitutes an highly important
  alternative determination of \mW, ... $t$-channel vs.\ $s$-channel,
  DIS vs.\ $ee$ and $pp$. $W$ mass not produced on-shell, but a
  measurement from purely virtual effects.
}
The LHeC will constitute the most precise measurement in DIS, where
presently the most precise measurement was obtained by
H1~\cite{Spiesberger:2018vki} 
($\mW(\text{H1})=80.520\pm0.115\,\GeV$). 

{\color{green}
The prospected measurement of the $W$-boson mass with LHeC data will be among the most
precise measurements and will significantly improve the world average
value of \mw, which is currently dominated by the ATLAS measurement~\cite{Aaboud:2017svj}.
}



% --------------------------------------------------------
%  Z mass
% --------------------------------------------------------
A determination of the $Z$-boson mass in the PDF+EW fit yields
expected experimental uncertainties of $\Delta\mZ=11\,MeV$ and 13\,MeV for
LHeC-60 and LHeC-50, respectively.
Altogether, these are of similar size than those of \mW.
The precision of \mZ\ cannot compete
with the precise measurements at the $Z$-pole by LEP+SLD, and future
$e^+e^-$ colliders may even further improve~\cite{todo,FCCee,ILC}.

% --------------------------------------------------------
%  W and Z mass
% --------------------------------------------------------
The result from a  simultaneous determination of \mW\ and \mZ\ is
displayed in Figure~\ref{fig:mWmZ}, where the 68\,\% confidence level
contours are displayed.
The precision of these two mass parameters in a simultaneous fit is
altogether only moderate. 
A more meaningful test of the high-energy behaviour of electroweak
theory at a much higher precision can be performed by imposing
additionally the precise measurement of $G_F$ as additional constraint.
As a result, a very shallow ellipse is obtained, which is due to the
high experimental precision of the $G_F$ measurement.
Such a fit simultaneously determines and tests the
behaviour of electroweak theory at high energies, while only
low-energy parameters $\alpha$ and $G_F$ are taken as inputs.

\begin{figure}[tbhp]
    \centering
    \includegraphics[width=0.42\textwidth]{plot_mWmZ_mitGf}
    \caption{
      Simultaneous determination of the top-quark mass $M_t$ and
      $W$-boson mass \mw\ from LHeC-60 or LHeC-50 data (left).
      The additional precision measurement of \gf from PSI yields a
      strong contraint and a very shallow ellipse is obtained.
      Simultaneous determination of the $W$-boson and $Z$-boson masses
      from LHeC-60 or LHeC-50 data (right).
    }
    \label{fig:mWmZ}
\end{figure}

{\color{magenta}Dicsussion about scheme dependence of parameters?}


\emph{The subleading contributions to the vertex and self-energy corrections
have a Higgs-boson mass dependence and are proportional to
$\log\tfrac{M^2_H}{\mw^2}$.
When fixing all other EW parameters the Higgs boson mass could be
constrained indirectly through these loop corrections
with a precision of $\Delta m_H=^{+29}_{-23}$ to $^{+24}_{-20}\,\GeV$
for different LHeC scenarios.}



%-----------------------------------------------------------------------
%   top quark mass
%-----------------------------------------------------------------------
\clearpage
\section{Top quark mass determination}
The inclusive DIS data are sensitive to the top-quark mass $M_t$
through radiative corrections and the $M_t$ dependent terms are the
dominant corrections to the vertex and to the propagator self-energies.
They are considered in the $\rho$ and $\kappa$ parameters and in the
correction factor $\Delta r$, and the leading contributions are
proportional to $M_t^2$. %$\gf{M_t^2}$.
This allows for an indirect determination of the top-quark mass using
LHeC inclusive DIS data, and a determination of $M_t$ will yield an
uncertainty of $\Delta M_t = 1.8\,\GeV$ to 2.2\,\GeV.
Assuming an uncorrelated uncertainty of the DIS pseudo-data of
$0.25\,\%$ the uncertainty of $M_t$ becomes as small as
\begin{equation}
  \Delta M_t=1.1 \text{~~to~~} 1.4\,\GeV
\end{equation}
for 60 and 50\,\GeV electron beams, respectively.
This would
%greatly improve over the limited precision obtained with the final H1
%data~\cite{Spiesberger:2018vki}, and would
represent a very precise
indirect determination of the top-quark mass from purely electroweak
corrections and thus being fully complementary to measurements
from real $t$-quark production, which suffer often from sizeable QCD corrections.
\begin{figure}[tbhp]
    \centering
    \includegraphics[width=0.42\textwidth]{plot_lhec_mWmt}
    \caption{
      Simultaneous determination of the $W$-boson and $Z$-boson masses
      from LHeC-60 or LHeC-50 data in comparision with the LEP+SLD
      combination and results from a global EW fit~\cite{Haller:2018nnx}.
    }
    \label{fig:mWmt}
\end{figure}

More generally, and to some extent dependent on the choice of the
renormalisation scheme, the leading self-energy and vertex corrections
are proportional to $\tfrac{M_t^2}{\mw^2}$ and thus a simultaneous
determination of $M_t$ and \mw\ is feasible.
The prospects for such a simultaneous determination of $M_t$ and \mw is
displayed in figure~\ref{fig:mWmt}.
It is remarkable, that the precision of the LHeC is
superior than the LEP combination~\cite{ALEPH:2005ab}, which combines results from all of the
four LEP experiments and the SLD experiment, and in an optimistic
scenario with $E_e=60\,\GeV$ and an uncorrelated uncertainty of
$0.25\,\%$ an uncertainty similar to the global electroweak
fit~\cite{Haller:2018nnx} can be achieved.
In a fit without PDF parameters similar uncertainties are found (not shown),
which illustrates that the determination of EW parameters is
to a large extent independent on the QCD phenomenology and the PDFs.

%-----------------------------------------------------------------------
%   Results: couplings
%-----------------------------------------------------------------------
\clearpage
\section{Weak Neutral Current Couplings}
The vector and axial-vector couplings of up-type and
down-type quarks to the $Z$, $g_V^{u/d}$ and $g_A^{u/d}$, see
eq.~\eqref{eq:gV-LO}) are determined
in a fit of the four coupling parameters together with the PDFs.
These fit-parameters are defined in the Born approximation and
\Qsq\ dependent higher-order corrections are calculated strictly in
the SM formalism in the 1-loop approximation.
The resulting uncertainties are collected in table~\ref{tab:couplings} and
compared to the current most precise values in figure~\ref{fig:couplings}.
%

\begin{figure}[tbhp]
    \centering
    \includegraphics[width=0.40\textwidth]{plot_couplings_u}
    \hskip0.05\textwidth
    \includegraphics[width=0.40\textwidth]{plot_couplings_d}
  \caption{
    Weak-neutral-current vector and axial-vector couplings of $u$-type quarks to the $Z$-boson
    (left), and those of the $d$-type quarks (right) at 68\,\% confidence level~(C.L.) for
    simulated LHeC data with $E_e=50\,\GeV$.
    The LHeC expectations are compared with results from the combined LEP experiments~\cite{ALEPH:2005ab}
    and single measurements by D0~\cite{D0} and H1~\cite{Spiesberger:2018vki}.
    The standard model expectations are diplayed by a red star.
  }
  \label{fig:couplings}
\end{figure}

\begin{table}[bth]
\footnotesize
%log.couplings.lhec60.uncorr.0.5.txt
%EPRC.au                   =          0.5   +/-   0.00220346
%EPRC.ad                   =         -0.5   +/-   0.00552205
%EPRC.vu                   =     0.202804   +/-   0.00152003
%EPRC.vd                   =    -0.351402   +/-   0.0046019
%log.couplings.lhec60.uncorr.0.25.txt
%EPRC.au                   =          0.5   +/-   0.00152328
%EPRC.ad                   =         -0.5   +/-   0.00342166
%EPRC.vu                   =     0.202804   +/-   0.00100496
%EPRC.vd                   =    -0.351402   +/-   0.0027548
%log.couplings.lhec50.uncorr.0.5.txt
%EPRC.au                   =          0.5   +/-   0.00354618
%EPRC.ad                   =         -0.5   +/-   0.00829618
%EPRC.vu                   =     0.202804   +/-   0.00278195
%EPRC.vd                   =    -0.351402   +/-   0.00673197
%
  \centering
  \begin{tabular}{cr@{\hskip4pt}lccc}
    \hline
    Coupling  &  \multicolumn{2}{c}{PDG} & \multicolumn{3}{c}{Expected uncertainties} \\
    parameter &  &  & LHeC-60 & LHeC-60 ({\footnotesize$\delta_\text{uncor.}=0.25\,\%$})  & LHeC-50 \\
    \hline
    $\au$  & $0.50$   & $^{+0.04}_{-0.05}$   & $0.0022$ & $0.0015$ & $0.0035$ \\
    $\ad$  & $-0.514$ & $^{+0.050}_{-0.029}$ & $0.0055$ & $0.0034$ & $0.0083$ \\
    $\vu$  & $0.18$   & $\pm0.05$            & $0.0015$ & $0.0010$ & $0.0028$ \\
    $\vd$  & $-0.35$  & $^{+0.05}_{-0.06}$   & $0.0046$ & $0.0027$ & $0.0067$ \\
\hline
  \end{tabular}
  \caption{
    Standard model expectations for the light-quark weak neutral
    couplings ($\au$,$\ad$,$\vu$,$\vd$) together with the currently
    most precise uncertainties (PDG~\cite{PDG}) and the prospected
    uncertainties for different LHeC scenarios.
    The LHeC prospects are obtained in a simultaneous fit of the PDF
    parameters and all four coupling parameters at a time.
     }
     \label{tab:couplings}
\end{table}
%


\begin{table}[bht]
\footnotesize
%log.couplings.lhec60.uncorr.0.5.txt
%EPRC.au                   =          0.5   +/-   0.00220346
%EPRC.ad                   =         -0.5   +/-   0.00552205
%EPRC.vu                   =     0.202804   +/-   0.00152003
%EPRC.vd                   =    -0.351402   +/-   0.0046019
%log.couplings.lhec60.uncorr.0.25.txt
%EPRC.au                   =          0.5   +/-   0.00152328
%EPRC.ad                   =         -0.5   +/-   0.00342166
%EPRC.vu                   =     0.202804   +/-   0.00100496
%EPRC.vd                   =    -0.351402   +/-   0.0027548
%log.couplings.lhec50.uncorr.0.5.txt
%EPRC.au                   =          0.5   +/-   0.00354618
%EPRC.ad                   =         -0.5   +/-   0.00829618
%EPRC.vu                   =     0.202804   +/-   0.00278195
%EPRC.vd                   =    -0.351402   +/-   0.00673197
%
  \centering
  \begin{tabular}{cr@{\hskip4pt}lccc}
    \hline
    Fit &   \multicolumn{2}{c}{PDG} & \multicolumn{3}{c}{Expected uncertainties} \\
    parameter &  &  & LHeC-60 & LHeC-60 ({\footnotesize$\delta_\text{uncor.}=0.25\,\%$})  & LHeC-50 \\
    \hline
    $\au$+PDF  & $0.50$   & $^{+0.04}_{-0.05}$   & $0.00xy$ & $0.00xy$ & $0.00xy$ \\
    $\ad$+PDF  & $-0.514$ & $^{+0.050}_{-0.029}$ & $0.00xy$ & $0.00xy$ & $0.00xy$ \\
    $\vu$+PDF  & $0.18$   & $\pm0.05$            & $0.00xy$ & $0.00xy$ & $0.00xy$ \\
    $\vd$+PDF  & $-0.35$  & $^{+0.05}_{-0.06}$   & $0.00xy$ & $0.00xy$ & $0.00xy$ \\
    $\ve$+PDF  & $ $      & $       $            & $0.00xy$ & $0.00xy$ & $0.00xy$ \\
    $\ve$+PDF  & $ $      & $                $   & $0.00xy$ & $0.00xy$ & $0.00xy$ \\
    \hline
    $\vu+\au$+PDF  &  \\
    $\vd+\ad$+PDF  &  \\
    $\ve+\gae$+PDF  &  \\
    \hline
  \end{tabular}
  \caption{
    {\color{red}Todo.\color{green} Highest achievable precision for a 1D.
    \color{red} Shall we include such a table?  What about the '2D'
    fits ?}
    \color{blue} Other tables are below.
  }
     \label{tab:couplings2}
\end{table}
%







The two-dimensional uncertainty contours at the
68\,\% confidence level obtained with LHeC with $E_e=50\,\GeV$ are
displayed in figure~\ref{fig:couplings} for the two quark families and
compared to recent measurements.
While the current determinations from $e^+e^-$, $ep$ or $p\bar{p}$
data have all somewhat similar precision,
the future LHeC data will greatly improve the
precision of the weak neutral-current couplings and
expected uncertainties are an order of magnitude smaller than the
currently most precise ones~\cite{PDG}.
An electron beam energy of $E_e=60\,\GeV$ or reduced
experimental uncertainties will of course further improve this
measurement.
Alternatively, these coupling parameters can be determined in a
simplified procedure, where only one or two coupling parameters are
determined simultaneously.
Such a procedure yields even higher precision and the prospected
uncertainties are collected in table~\ref{tab:couplings2}

{\color{magenta} Discussion about \ve\ and \gae.}





%-----------------------------------------------------------------------
%   Results: form factors
%-----------------------------------------------------------------------
\clearpage
\section{Determination of the $\rho$ and $\kappa$ parameters}
Beyond the born approximation the weak couplings are
subject to higher-order loop corrections.
These corrections are commonly parameterised into the
$\rho_\text{NC}$, $\kappa_\text{NC}$ and $\rho_\text{CC}$
form factors and
many extensions of the Standard Model predict modifications of these
factors, thus making them precision observables.


\subsection{Anomalous NC form factors} % 'anomalous modifications of the NC form factor'?
In NC interactions universal higher-order corrections to the vector
and axial-vector couplings are be taken into account by $\Qsq$
dependent form factors $\rho_{\text{NC}}$ and $\kappa_{\text{NC}}$.
%
Subsequently, commonly the so-called effective weak mixing
angle at the $Z$-pole,
$\sin\theta_{\text{W}}^{\text{eff},q/f}:=\kappa_{\text{NC}, q/f}\ \sw$
is studied with real data, since it is well accessible in asymmetry
measurements in $e^+e^-$ collisions.
%
In inclusive DIS, however, this parameter can be determined only
together with the $\rho$ parameter due to the \Qsq\ dependence and the
presence of the photon exchange terms.

Many of the extensions of the SM predict modifications of the weak
neutral-current couplings and thus modifications of the
$\rho_{\text{NC}}$ and $\kappa_{\text{NC}}$.
Such modifications can be described conveniently by introducing the anomalous parameters \rhop\ and \kapp{},
which can be also considered to be \Qsq\ dependent:
\begin{align}
  g_A^f &= \sqrt{\rho_{\text{NC}, f}\rhop{,f}} \Itf
  \label{eq:gA} \, ,
  \\
  g_V^f &= \sqrt{\rho_{\text{NC}, f}\rhop{,f}} \left( \Itf - 2 Q_f \kappa_{\text{NC}, f}\kapp{,f}\sw \right)
  \label{eq:gV} \,.
\end{align}
Consequently, the estimated relative uncertainties of these parameters
can be interpreted as the relative uncertainty of a
direct determination of $\rho_{\rm NC}$ or $\sin^2\theta_w^{\rm eff}$.
In the following, we compare the uncertainties from LHeC pseudo-data
with the relative uncertainties of
$\sin\theta_{\text{W}}^{\text{eff}}$ and the $\rho_\text{NC}$
parameter from the LEP+SLD combination~\cite{ALEPH:2005ab}, while
setting their actual value to the SM expectation, since we are only
interested in the sensitivity to those parameters at the moment.



First, the uncertainties the anomalous form factors $\rho_\text{NC}^\prime$
and $\kappa_\text{NC}^\prime$ is obtained in a fit together with the PDFs and is
displayed in fig.~\ref{fig:rhokappa} (left).
The results are compared with results of the LEP+SLD combinations and
found to have a similar uncertainty of about a few permille.
\begin{figure}[thbp]
    \centering
    \includegraphics[width=0.32\textwidth]{plot_NC_rhokap}
    \hskip0.02\textwidth
    \includegraphics[width=0.32\textwidth]{plot_NC_rhokap_u}
    \includegraphics[width=0.32\textwidth]{plot_NC_rhokap_d}
  \caption{
    Expectations at the 68\,\% confidence level for the determination
    of the $\rho^\prime$ and $\kappa^\prime$
    parameters assuming single anomalous factors for all fermions
    (left). The results for three different LHeC scenarios are
    compared with the achieved uncertainties from the LEP+SLD
    combination~\cite{ALEPH:2005ab} for the determination the
    respective factors for leptons.
    Right: uncertainties for the simultaneous determination of the
    anomalous form factors for $u$ and $d$-type quarks, assuming known
    values for the electron parameters.
    The values are compared with uncertainties reported by LEP+SLD for
    the determination of the values $\rho_{\text{NC},(c,b)}$ and
    $\sin\theta_{\text{W}}^{\text{eff},(c,b)}$ for charm or bottom quarks,
    respectively.
  }
  \label{fig:rhokappa}
\end{figure}



Secondly, when assuming the weak couplings of the electron to be given by the SM formalism,
the anomalous form factors for the two quark families can be
determined, see fig.~\ref{fig:rhokappa} (right).
This is done in a fit of the four parameters
and thus the results can be translated to the results of the coupling
fits (see tab.~\ref{tab:couplings} and fig.~\ref{fig:couplings}).
In fig.~\ref{fig:rhokappa} the uncertainties for three different LHeC
scenarios are compared with uncertainties from the LEP+SLD
combination~\cite{ALEPH:2005ab}.
Since in the LEP+SLD analysis the values of $\rho$ and $\kappa\sw$
are determined themselves, we compare only the size of the uncertainties.
Furthermore it shall be noted, that LEP is mainly sensitive to the
parameters of leptons or heavy quarks, while LHeC data is more
sensitive to light quarks ($u$,$d$,$s$), and thus the LHeC
measurements are highly complementary.


%\paragraph{Prospects for neutral current parameters}


{\color{blue} Maximum sensitivity is obtained in 1-parameter fits, and prospects are collected in table:~\ref{tab:rhopwithcorrelations}}.
\begin{table}[bhtp]
  \footnotesize
  \centering
  \begin{tabular}{lcr@{$\,\pm\,$}lr@{$\,\pm\,$}l}
    \hline
    Fit parameters & Parameter & \multicolumn{2}{c}{LHeC}& \multicolumn{2}{c}{FCC}  \\
    \hline
    \rhopu+PDF & \rhopu &  1  &  0.009    & 1  & 0.004  \\
    \kappu+PDF & \kappu &  1  &  0.004    & 1  & 0.003  \\
    \rhopd+PDF & \rhopd &  1  &  0.014    & 1  & 0.006  \\
    \kappd+PDF & \kappd &  1  &  0.023    & 1  & 0.013  \\
    \rhope+PDF & \rhope &  1  &  0.006    & 1  & 0.003  \\
    \kappe+PDF & \kappe &  1  &  0.003    & 1  & 0.002  \\
    \hline
    \rhopq+PDF & \rhopq &  1  &  0.0059    & 1  & 0.0027  \\
    \kappq+PDF & \kappq &  1  &  0.0038    & 1  & 0.0024  \\
    \hline
    \rhopf+PDF & \rhopf &  1  &  0.0031    & 1  & 0.0015  \\ %0.00145641
    \kappf+PDF & \kappf &  1  &  0.0019    & 1  & 0.0011  \\
    \hline
    \\
    \multicolumn{3}{l}{Expectations for \rhopW{,f} (CC)} \\
    \hline
    % log.final.PAR19eq50.3p.PDF.2.txt
    \rhopW{,f}+PDF   & \rhopW{,f}                &  1  &   0.0043  & 1  & 0.0027 \\
    \rhopW{,eq}+PDF  & \rhopW{,eq}               &  1  &   0.0027  & 1  &  0.0011\\
    \rhopW{,e\bar{q}}+PDF  & \rhopW{,e\bar{q}}   &  1  &   0.0030  & 1  &  0.0012\\
    \hline
  \end{tabular}
  \caption{
    {\color{red} note: these numbers need to be updated, if we want to
    include such a table.}
    Results for $\rhop{}$, $\kapp{}$ and \rhopW{} parameters.
  }
  \label{tab:rhopwithcorrelations}
\end{table}



\begin{figure}[thbp]
    \centering
    \includegraphics[width=0.40\textwidth]{plot_rhopQ2}
    \hskip0.02\textwidth
    \includegraphics[width=0.40\textwidth]{plot_kapQ2}
    \caption{
      Scale dependence of the anomalous modification of the $\rho$ and
      $\kappa$ parameters for two different LHeC scenarios. In case of
      LHeC-60, with $E_e=60\,\GeV$, an uncorrelated uncertainty of
      $0.25\,\%$ on the data points is imposed.
  }
  \label{fig:rhokappaQ2}
\end{figure}
A meaningful test of the SM is obtained by determining the scale
dependence of the anomalous form factors.
In case of the $\kappa_\text{NC}^\prime$ parameter this procedure is equivalent to the
running of the effective weak mixing angle,
$\sin\theta_{\text{W}}^{\text{eff}}(\mu)$.
However, DIS is again complementary to other measurements since the
process is mediated in the space-like regime, i.e.\ $\Qsq=-q^2$ with
$q$ being the boson four-momentum.
Prospects for a determination of $\rho_\text{NC}^\prime$ or $\kappa_\text{NC}^\prime$ at
different \Qsq\ values is displayed in fig.~\ref{fig:rhokappaQ2} and
compared to results obtaind by H1.
The value of $\kappa_\text{NC}^\prime(\mu)$ can be easily translated to a
measurement of the effective weak mixing angle and shows, that this
quantity can be determind with a precision of up to 0.1\,\% and with
bettern than at least 1\,\% over a wide kinematic range of about
$25<\sqrt{\Qsq}<700\,\GeV$.



%% EW in CC
\subsection{Electroweak effects in charged-current exchange}
At the LHeC, the charged current sector can be uniquely measured
over many orders of magnitude in $Q^2$, due to the excellent tracking
detectors, calorimetry, and high-bandwidth triggers.
%
%This provides a precise determination of the $W$-boson mass (see
%above)
%and furthermore, the form factors of the
%effective couplings of the fermions to the $W$-boson can be uniquely
%measured.
%
Similarly as in the NC case, higher-order EW corrections to the CC
cross sections are collected in the effective couplings of the
fermions to the $W$-boson and these are represented in terms of the
form factors $\rho_{\text{CC},eq}$ and $\rho_{\text{CC},e\bar{q}}$.
%
In the SM formalism, only two of these form factors are present.
Since extensions of the SM formalism likely alter these factors, we
introduce two anomalous form factors
$\rho^\prime_{\text{CC},(eq/e\bar{q})}$,
whose values are unity for the strict validity of the SM and
%$\rho_{\text{CC},(eq/e\bar{q})} \rightarrow \rho^\prime_{\text{CC},(eq/e\bar{q})} \rho_{\text{CC},(eq/e\bar{q})}$.
the CC structure functions then become 
\begin{align}
  W_2^- &=
  x \left( (\rho_{\text{CC}, eq}\rhopW{,eq})^2 U + (\rho_{\text{CC},e\bar{q}}\rhopW{,e\bar{q}})^2 \overline{D} \right)
  \, ,
  \\
  xW_3^- &=
  x \left( (\rho_{\text{CC},eq}\rhopW{,eq})^2 U - (\rho_{\text{CC},e\bar{q}}\rhopW{,e\bar{q}})^2 \overline{D} \right)
  \, ,
\\
  W_2^+ &=
  x \left( (\rho_{\text{CC},eq}\rhopW{,eq})^2 \overline{U}+ (\rho_{\text{CC},e\bar{q}}\rhopW{,e\bar{q}})^2 D \right)
  \, ,
  \\
  xW_3^+ &=
  x \left( (\rho_{\text{CC},e\bar{q}}\rhopW{,e\bar{q}})^2 D - (\rho_{\text{CC},eq}\rhopW{,eq})^2 \overline{U} \right)
  \, .
\end{align}



\begin{figure}[thbp]
    \centering
    \includegraphics[width=0.40\textwidth]{plot_rhoWellipse} 
   \hskip0.05\textwidth
    \includegraphics[width=0.40\textwidth]{plot_rhopWQ2}
  \caption{
    Left: anomalous modifications of the charged current form factors
    $\rho^\prime_{\text{CC},eq}$ and $\rho^\prime_{\text{CC},e\bar{q}}$ for
    different LHeC scenarios in comparison with the H1
    measurement~\cite{Spiesberger:2018vki}.
    Right: scale dependent measurement of the anomalous modification
    of the charged current form factor $\rho^\prime_{\text{CC}}(\Qsq)$, assuming $\rho^\prime_{\text{CC},eq}=\rho^\prime_{\text{CC},e\bar{q}}=\rho^\prime_{\text{CC}}$.
  }
  \label{fig:rhoCC}
\end{figure}
The prospects for a determination of these anomalous form factors with
LHeC data are obtained by performing a fit of the two parameters
$\rho^\prime_{\text{CC},eq}$ and $\rho^\prime_{\text{CC},e\bar{q}}$ together with
the PDFs.
This procedure is equivalent to a direct determination of the the
factors $\rho_{\text{CC},eq}$ and $\rho_{\text{CC},e\bar{q}}$
themselves and results are displayed in fig.~\ref{fig:rhoCC} and
collected in table~\ref{tab:rhoCC}.
It is found, that these parameter can be determined with a relative
uncertainty of up to 0.2--0.3\,\%.
This represents a unique measurement at very high precision of charged
current interactions at considerably high scales.


Furthermore, the \Qsq dependence of the anomalous form factors can be studied.
Relative uncertainties obtained in a determination of
$\rho^\prime_\text{CC}$ at various values of \Qsq, where we used
$\rho^\prime_{\text{CC},eq}=\rho^\prime_{\text{CC},e\bar{q}}$ and
in that fit all parameters at the various \Qsq values are fitted
simultaneously and together with the PDFs, are displayed in fig.~\ref{fig:rhoCC}.
It is found, that with the LHeC data the potential deviation of
$\rho^\prime_\text{CC}$ from
the SM expectation can be studied with very high precision and over a
large range in $\sqrt{\Qsq}$, up to values of $\sqrt{\Qsq}$ of about 500\,\GeV.
\begin{table}[bhtp]
  \footnotesize
  \centering
  \begin{tabular}{lcr@{$\,\pm\,$}lr@{$\,\pm\,$}l}
    \hline
    Fit parameters & Parameter & \multicolumn{2}{c}{LHeC}& \multicolumn{2}{c}{FCC}  \\
    \hline
    \rhopW{,eq}+\rhopW{,e\bar{q}}+PDF  & \rhopW{,eq}               &  1  &   0.0027  & 1  &  0.0011\\
    \rhopW{,eq}+\rhopW{,e\bar{q}}+PDF  & \rhopW{,e\bar{q}}         & 1  &   0.0030  & 1  &  0.0012\\
    
    \hline
    % log.final.PAR19eq50.3p.PDF.2.txt
    \rhopW{,f}+PDF   & \rhopW{,f}                &  1  &   0.0043  & 1  & 0.0027 \\
    \rhopW{,eq}+PDF  & \rhopW{,eq}               &  1  &   0.0027  & 1  &  0.0011\\
    \rhopW{,e\bar{q}}+PDF  & \rhopW{,e\bar{q}}   &  1  &   0.0030  & 1  &  0.0012\\
    \hline
  \end{tabular}
  \caption{
    Expectations for \rhopW{,f} (CC).
    {\color{red} note: these numbers need to be updated, if we want to
    include such a table.}
    Results for $\rhop{}$, $\kapp{}$ and \rhopW{} parameters.
  }
  \label{tab:rhoCC}
\end{table}



%-----------------------------------------------------------------------
%   Results: sin2thw
%-----------------------------------------------------------------------
\clearpage
\section{\boldmath The effective leptonic weak mixing angle $\sin^2\theta_\textrm{W,eff}^{ll}$}


%-----------------------------------------------------------------------
%   EW effects in PDF fits
%-----------------------------------------------------------------------
\clearpage
\section{Impact of electroweak effects on the determination of PDFs}
???


%-----------------------------------------------------------------------
%   FCC
%-----------------------------------------------------------------------
\clearpage
\section{Prospects for the FCC-eh}
In this section, we briefly outline the prospects for the FCC-eh and
give numbers for a few benchmark parameters.




%-----------------------------------------------------------------------
%                            Summary
%-----------------------------------------------------------------------
\clearpage
\section{Summary and Conclusion}
\label{sect:Conclusion}
{\color{green} Text partially equivalent to the FCC-CDR}
%
Simulated neutral current and charged current inclusive DIS cross
sections at the LHeC are explored for a determination of the
fundamental parameters of electroweak theory and for precision tests
of the Standard Model formalism.
%
The high center-of-mass energy and the large integrated
luminosity at the LHeC will allow for the first time precision electroweak
measurement in DIS at high scales.
%


Highest precision is achieved for the measurement of the $W$-boson
mass with a prospected experimental uncertainty of up to
$\Delta\mW=5\,\MeV$, and an outstanding  
precision can also be achieved for the weak neutral current couplings of
light quarks to the $Z$ boson.
%
The space-like momentum transfer in DIS further allows for unique
scale dependent test of the electroweak theory, both, for neutral and
charged currents.
This includes for instance scale dependent measurements of the effective
weak mixing angle in the range of about $40<\sqrt{\Qsq}<700\,\GeV$
with a precision up to a few permille.

Further direct measurements, as for instance Higgs production,
top-quark production or single $W$ or $Z$
production cross sections, or vector-boson-scattering cross sections,
as well as heavy flavor cross sections in NC and CC DIS, will provide further
improvements for electroweak precision measurements.

The measurements will not be limited by the need for parton
distribution functions.
In many cases, the measurements are complementary to measurements in
$e^+e^-$ or hadron-hadron collisions.


In summary, the inclusive NC and CC DIS cross sections measured at
LHeC provide a unique opportunity to perform high-precision
determinations of fundamental electroweak parameters and test the
quantum  nature of electroweak processes with high precision.
Complementarity!

Unique measurements will be made for weak charged currents and
for the scale-dependence of electroweak interactions.




%-----------------------------------------------------------------------
%                          Acknowledgements
%-----------------------------------------------------------------------
\section*{Acknowledgements}
Acknowledgements: Z.~Zhang, A.~Sch\"oning, M.~Klein, S.~Schmitt



%-----------------------------------------------------------------------
%\input{fcc_draft}


%-----------------------------------------------------------------------
%   Results from LHeC inclusive DIS data
%-----------------------------------------------------------------------
\clearpage
\section{Tables}
\subsection{Determinations of single parameters}
\begin{table}[h]
  \footnotesize
  \centering
  \begin{tabular}{lcr@{$\,\pm\,$}lr@{$\,\pm\,$}l}
    \hline
    Fit parameters & Parameter & \multicolumn{2}{c}{LHeC}& \multicolumn{2}{c}{FCC}  \\
    \hline
    % log.fcc.6p.u.d.e.txt
    % EPRC.rhopu                =            1   +/-   0.00864794
    % EPRC.zkapu                =            1   +/-   0.00886882
    % EPRC.rhopd                =            1   +/-   0.028591
    % EPRC.zkapd                =            1   +/-   0.0438381
    % EPRC.rhope                =            1   +/-   0.0136914
    % EPRC.zkape                =            1   +/-   0.00510635
    \rhopd+\kappd+\rhopu+\kappu+\rhop{,e}+\kapp{,e}+PDF
    & \rhopu &  1  &  0.031    & 1  & 0.009 \\
    & \kappu &  1  &  0.013    & 1  & 0.009 \\
    & \rhopd &  1  &  0.062    & 1  & 0.029   \\
    & \kappd &  1  &  0.076    & 1  & 0.044  \\
    & \rhope &  1  &  0.036    & 1  & 0.014  \\
    & \kappe &  1  &  0.008    & 1  & 0.005 \\
    \hline
    % /nfs/dust/h1/group/britzger/alpos/Alpos/../log.fcc.4p.u.d.txt
    % EPRC.rhopu                =            1   +/-   0.00748481
    % EPRC.zkapu                =            1   +/-   0.00663437
    % EPRC.rhopd                =            1   +/-   0.0159018
    % EPRC.zkapd                =            1   +/-   0.0419873
    %lhec
    % EPRC.rhopu                =            1   +/-   0.0185425
    % EPRC.zkapu                =            1   +/-   0.0102184
    % EPRC.rhopd                =            1   +/-   0.0394555
    % EPRC.zkapd                =            1   +/-   0.0757502
    \rhopd+\kappd+\rhopu+\kappu+PDF
    & \rhopu &  1  &  0.019  & 1  & 0.007   \\
    & \kappu &  1  &  0.010  & 1  & 0.007   \\
    & \rhopd &  1  &  0.039  & 1  & 0.016   \\
    & \kappd &  1  &  0.076  & 1  & 0.042   \\
    \hline
    % /nfs/dust/h1/group/britzger/alpos/Alpos/../log.fcc.4p.e.q.txt
    %rhopq                     =            1   +/-   0.00827489
    %zkapq                     =            1   +/-   0.005199
    %EPRC.rhope                =            1   +/-   0.0100664
    %EPRC.zkape                =            1   +/-   0.00504328
    \rhop{,q}+\kapp{,q}+\rhop{,e}+\kapp{,e}+PDF
    & \rhop{,q} & 1 &  0.029  & 1 &  0.008  \\
    & \kapp{,q} & 1 &  0.007  & 1 &  0.005 \\
    & \rhop{,e} & 1 &  0.032  & 1 &  0.010 \\
    & \kapp{,e} & 1 &  0.008  & 1 &  0.005 \\
    \hline
    % /nfs/dust/h1/group/britzger/alpos/Alpos/../log.fcc.4p.e.q.txt
    %rhopq                     =            1   +/-   0.00827489
    %zkapq                     =            1   +/-   0.005199
    %EPRC.rhope                =            1   +/-   0.0100664
    %EPRC.zkape                =            1   +/-   0.00504328
    \rhop{,q}+\kapp{,q}+\rhop{,e}+\kapp{,e}+PDF
    & \rhop{,q} & 1 &  0.029  & 1 &  0.008  \\
    & \kapp{,q} & 1 &  0.007  & 1 &  0.005 \\
    & \rhop{,e} & 1 &  0.032  & 1 &  0.010 \\
    & \kapp{,e} & 1 &  0.008  & 1 &  0.005 \\
    \hline
    % /nfs/dust/h1/group/britzger/alpos/Alpos/
    \rhopu+\kappu+PDF
    & \rhopu &  1  &  0.011    & 1 & 0.005\\
    & \kappu &  1  &  0.005    & 1 & 0.003\\
    \hline
    % /nfs/dust/h1/group/britzger/alpos/Alpos/
    \rhopd+\kappd+PDF
    & \rhopd &  1  &  0.022    &  1  & 0.011 \\
    & \kappd &  1  &  0.038    &  1  & 0.021 \\
    \hline
    \rhop{,e}+\kapp{,e}+PDF
    & \rhope &  1  & 0.009      & 1  & 0.005 \\
    & \kappe &  1  & 0.005      & 1  & 0.003  \\
    \hline
    \rhop{,f}+\kapp{,f}+PDF
    & \rhope &  1  & 0.0042      & 1  & 0.0022 \\
    & \kappe &  1  & 0.0026      & 1  & 0.0016  \\
    \hline
    \\
    \multicolumn{3}{l}{Fits including \rhopW{,f} (CC)} \\
    \hline
    % log.final.PAR19eq50.3p.PDF.2.txt
    \rhop{,f}$+$\kapp{,f}$+$\rhopW{,f}+PDF
    & \rhop{f}   &  1  &  0.0045   & 1  & 0.0021 \\
    & \kapp{f}   &  1  &  0.0027   & 1  & 0.0015 \\
    & \rhopW{,f} &  1  &  0.0043   & 1  & 0.0027 \\
    % & $\rhop{W,f}=1.002\pm0.008$ & \\
    \hline
  \end{tabular}
  \caption{
    Results for $\rhop{}$, $\kapp{}$ and \rhopW{} parameters, and their
    correlation coefficients.
  }
  \label{tab:rhopwithcorrelations}
\end{table}


\begin{table}[h]
  \footnotesize
  \centering
  \begin{tabular}{lr@{$\,=\,$}c@{$\,\pm\,$}l|cccccc}
    \hline
    Fit parameters & \multicolumn{3}{c}{Result} & \multicolumn{6}{l}{Correlation} \\
    \hline
    % /nfs/dust/h1/group/britzger/alpos/Alpos/
    \rhopd+\kappd+\rhopu+\kappu+\rhop{,e}+\kapp{,e}+PDF
    & \rhopu &    &      & 1.00 \\
    & \kappu &    &      & 0.  & 1.00 \\
    & \rhopd &    &      &$    $& $   $ & 1.00 \\
    & \kappd &    &      &$    $& $   $ &  & 1.00 \\
    & \rhope &    &      &      &       &  &      &  1.00 \\
    & \kappe &    &      & 0.   &       &  &      &  & 1.00 \\
    \hline
    % /nfs/dust/h1/group/britzger/alpos/Alpos/
    \rhopd+\kappd+\rhopu+\kappu+PDF
    & \rhopu &    &      & 1.00 \\
    & \kappu &    &      & 0.  & 1.00 \\
    & \rhopd &    &      &$    $& $   $ & 1.00 \\
    & \kappd &    &      &$    $& $   $ &  & 1.00 \\
    \hline
    % /nfs/dust/h1/group/britzger/alpos/Alpos/
    \rhop{,q}+\kapp{,q}+\rhop{,e}+\kapp{,e}+PDF
    & \rhop{,e} &    &     & 1.00 \\
    & \kapp{,e} &    &     & 0.  & 1.00 \\
    & \rhop{,q} &    &     &     & 0.  & 1.00 \\
    & \kapp{,q} &    &     & 0.  & $0$ & 0.0  & 1.00 \\
    \hline
    % /nfs/dust/h1/group/britzger/alpos/Alpos/
    \rhopu+\kappu+PDF
    & \rhopu &    &      & 1.00 \\
    & \kappu &    &      & 0.  & 1.00 \\
    \hline
    % /nfs/dust/h1/group/britzger/alpos/Alpos/
    \rhopd+\kappd+PDF
    & \rhopd &    &      & 1.00 \\
    & \kappd &    &      &  & 1.00 \\
    \hline
    % /nfs/dust/h1/group/britzger/alpos/Alpos/
    \rhop{,e}+\kapp{,e}+PDF
    & \rhope &    &      &  1.00 \\
    & \kappe &    &      &  & 1.00 \\
    \hline
    \\
    \multicolumn{3}{l}{Fits including \rhopW{,f} (CC)} \\
    \hline
    % log.final.PAR19eq50.3p.PDF.2.txt
    \rhop{,f}$+$\kapp{,f}$+$\rhopW{,f}+PDF
    & \rhop{f}   &    &     & 1.00 & \\
    & \kapp{f}   &    &     & 0. & 1.00  \\
    & \rhopW{,f} &    &     & 0. & $0.$ & 1.00\\
    % & $\rhop{W,f}=1.002\pm0.008$ & \\
    \hline
  \end{tabular}
  \caption{
    Results for $\rhop{}$, $\kapp{}$ and \rhopW{} parameters, and their
    correlation coefficients.
  }
  \label{tab:rhopwithcorrelations}
\end{table}


% ------------------------------------------------------------------------
%
% ------------------------------------------------------------------------
\clearpage
\section*{Useless stuff}
%Reasonable examples are determinations of (all) the weak neutral current
%couplings, $g_A^{f}$ and $g_V^{f}$ with $f=(e,u,d,s,c,b,t)$,
%or more general, directly the $\rho_{\text{NC}, f}$ and
%$\kappa_{\text{NC}, f}$ parameters.
These can be done either in the scale-indpendent LO approximation, by
taking the scale-dependent loop corrections from theory, or their
effective values can be determined at different scales.
More commonly though, the effective weak mixing
angle $\sin\theta_{\text{W}}^{\text{eff},f}:=\kappa_{\text{NC}, q/e}\ \sw$
is tested at different scales, while also this quantity has an
important scheme dependent component (see Ref.~\ref{pdg18} for a
concise dicsussion).
Noteworthy, most of the precision measurements are performed in the
time-like domain, i.e.\ for $\mu^2>0$, whereas DIS is mediated by space-like
momentum transfer, $\Qsq=-q^2$, and thus $q^2<0$.
In DIS at the LHeC, in addition, a large kinematic range can be tested.

Complementary, the measurement of the weak boson masses yields an interesting testing
case of the theory, in particular the measurement of \mw.
This is because \mw\ can be considered as input parameter to the
formalism, or alternatively, if the precision measurement of \gf\ is
been taken as input~\cite{Tishchenko:2012ie}, then \mw becomes a
prediction and can be confronted with the measurement directly.
