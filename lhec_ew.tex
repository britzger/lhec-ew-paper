% ================================================================
% LaTeX file with prefered layout for H1 paper drafts
% use: dvips -D600 file-name
% ================================================================
%%%%%%%%%%%%% LATEX HEADER
%%%%%%%%%%%%% DO NOT DELET NOT CHANGE ANYTHING IN THE HEADER
%%%%%%%%%%%%% UNLESS CLEARLY RECOMMENDED

\documentclass[12pt]{article}
%\usepackage[T1]{fontenc}
\usepackage[utf8]{inputenc}
%\bibliographystyle{desy19-080}
\bibliographystyle{utphys}
 %Choose a bibliograhpic style
\usepackage{amsmath}
\usepackage{amssymb}
%\usepackage{times}
\usepackage{txfonts}
\usepackage[mathlines,displaymath]{lineno}
\DeclareMathAlphabet{\mathbold}{OML}{txr}{b}{it}
%\usepackage[font={small,it}]{caption}
\usepackage[font={small}]{caption}

\usepackage{multirow}
%\usepackage{units}
%\usepackage{hhline}
\usepackage{dsfont}
\usepackage{xcolor}
%\usepackage{subcaption}
%\usepackage{pdflscape}
\usepackage{graphicx}
\usepackage{rotating}
\usepackage{paralist} % compactitem

%\usepackage{longtable}
\usepackage{xspace}
%\usepackage{bm} % 'bold' math symbols (better use \usepackage{newtxtext,newtxmath}, if available on latex distribution)
\usepackage{acronym}
\usepackage{dcolumn}
\newcolumntype{L}{D{.}{.}{2,5}}
%\usepackage{bbold}

%%%%%%%%%%%%% Comment the next two lines to remove the line numbering
%\usepackage[]{lineno}
%\linenumbers
%%%%%%%%%%%%%%

\newcommand{\includegraphicss}[2][]{\fbox{\includegraphics[#1]{#2}}}
%\newcommand{\includegraphicss}[2][]{\includegraphics[#1]{#2}}



\usepackage{hyperref} % has to be last package loaded
\hypersetup{colorlinks=true, urlcolor=blue}
%\usepackage{cite} % Use this package to display references as [21-25], which in contrast removes the hyperlink
\usepackage{cite} % DB: enable also clickable references (must be loaded after hyperref)
\hypersetup{
  colorlinks,
  citecolor=blue,
  linkcolor=red,
  urlcolor=blue
  }
  
%%%%%%%%%%%%
\renewcommand{\topfraction}{1.0}
\renewcommand{\bottomfraction}{1.0}
\renewcommand{\textfraction}{0.0}
\renewcommand{\arraystretch}{1.25} % make lines a bit larger for tables
\newlength{\dinwidth}
\newlength{\dinmargin}
\setlength{\dinwidth}{21.0cm}
\textheight23.5cm \textwidth16.0cm
\setlength{\dinmargin}{\dinwidth}
\setlength{\unitlength}{1mm}
\addtolength{\dinmargin}{-\textwidth}
\setlength{\dinmargin}{0.5\dinmargin}
\oddsidemargin -1.0in
\addtolength{\oddsidemargin}{\dinmargin}
\setlength{\evensidemargin}{\oddsidemargin}
\setlength{\marginparwidth}{0.9\dinmargin}
\marginparsep 8pt \marginparpush 5pt
\topmargin -42pt
\headheight 12pt
\headsep 30pt \footskip 24pt
\parskip 3mm plus 2mm minus 2mm

% do not indent first line of paragraph!
%\setlength{\parindent}{0pt}
\usepackage{parskip}
\hyphenation{pa-ra-me-ters}


%%%%%%%%%%%%%%%% END OF LATEX HEADER
%===============================title page=============================
\begin{document}  

%Roman numbers
\makeatletter
\newcommand*{\rom}[1]{\expandafter\@slowromancap\romannumeral #1@}
\makeatother

%% uncertainties: Absolute and relative uncertainties
\newcommand{\delrel}{\ensuremath{\delta}}
\newcommand{\delabs}{\ensuremath{\Delta}}
\newcommand{\dabs}[2][1]{\ensuremath{\delabs^{{\rm{#1}}}_{{#2}}}}
\newcommand{\drel}[2][1]{\ensuremath{\delrel^{{\rm{#1}}}_{{#2}}}}
\newcommand{\TODO}{\color{red}TODO\xspace}

\newcommand{\muf}{\ensuremath{\mu_{f}}\xspace}
\newcommand{\mur}{\ensuremath{\mu_{r}}\xspace}
\newcommand{\as}{\ensuremath{\alpha_s}\xspace}
\newcommand{\asmz}{\ensuremath{\alpha_s(M_Z)}\xspace}
\newcommand{\asmur}{\ensuremath{\alpha_s(\mur)}\xspace}
\newcommand{\aem}{\ensuremath{\alpha_{\mathrm{em}}}\xspace}
\newcommand{\Lumi}{\ensuremath{\mathcal{L}}}
\newcommand{\pb}{\rm pb}
\newcommand{\invpb}{\ensuremath{\rm{pb}^{-1}}}
\newcommand{\PDF}{\ensuremath{{\rm PDF}}\xspace}

%% observables
\newcommand{\xbj}{\ensuremath{x}\xspace}
\newcommand{\Qsq}{\ensuremath{Q^2}\xspace}

%% ep scattering
\newcommand{\ep}{\ensuremath{e^+}\xspace}
\newcommand{\emm}{\ensuremath{e^-}\xspace}
\newcommand{\epm}{\ensuremath{e^\pm}\xspace}

%% tex
\newcommand{\etal}{{\it et al.}}
\newcommand{\eq}{equation}
\newcommand{\fig}{figure}
\newcommand{\tab}{table}

%% units
\newcommand{\MeV}{\ensuremath{\mathrm{MeV}}\xspace}
\newcommand{\GeV}{\ensuremath{\mathrm{GeV}}\xspace}
\newcommand{\GeVsq}{\ensuremath{\mathrm{GeV}^2}\xspace}
\newcommand{\Ord}{\ensuremath{\mathcal{O}}}

%% EW parameters
\newcommand{\sw}{\ensuremath{{\rm sin}^2\theta_W}}
\newcommand{\dr}{\ensuremath{\Delta r}}
\newcommand{\gf}{\ensuremath{G_{\rm F}}}

\newcommand{\rhopW}[2][]{\ensuremath{\rho^{\prime}_{\text{CC}#2}}}
\newcommand{\rhop}[2][] {\ensuremath{\rho^{\prime}_{\text{NC}#2}}}
\newcommand{\kapp}[2][] {\ensuremath{\kappa^{\prime}_{\text{NC}#2}}}
\newcommand{\rhopu}{\rhop{, u}}
\newcommand{\kappu}{\kapp{, u}}
\newcommand{\rhopd}{\rhop{, d}}
\newcommand{\kappd}{\kapp{, d}}
\newcommand{\rhope}{\rhop{, e}}
\newcommand{\kappe}{\kapp{, e}}
\newcommand{\kapz}{\ensuremath{\kappa_{\text{NC}, f}}}
\newcommand{\rhoz}{\ensuremath{\rho_{\text{NC},f}}}
\newcommand{\Itf}{\ensuremath{I^3_{{\rm L},f}}}
\newcommand{\Itq}{\ensuremath{I^3_{{\rm L},q}}}


% couplings
\newcommand{\ad} {\ensuremath{g_A^d}}
\newcommand{\vd} {\ensuremath{g_V^d}}
\newcommand{\au} {\ensuremath{g_A^u}}
\newcommand{\vu} {\ensuremath{g_V^u}}
\newcommand{\aq} {\ensuremath{g_A^q}}
\newcommand{\vq} {\ensuremath{g_V^q}}
\newcommand{\gae}{\ensuremath{g_A^e}}
\newcommand{\ve} {\ensuremath{g_V^e}}


%% masses
\newcommand{\mt} {\ensuremath{m_t}}
\newcommand{\mW} {\ensuremath{m_W}}
\newcommand{\mWprop} {\ensuremath{m^{\rm prop}_W}}
\newcommand{\mWGfW} {\ensuremath{m^{(\gf,\mW)}_W}}
\newcommand{\mZ} {\ensuremath{m_Z}}
\newcommand{\mH} {\ensuremath{m_H}}


% Journal macro
\def\Journal#1#2#3#4{{#1}~{\bf #2} (#3) #4}
%
\def\NPB{Nucl. Phys.~}
\def\PRL{Phys. Rev. Lett.~}
\def\EPJC{Eur. Phys. J.~}
\def\PLB{Phys. Lett.~}
\def\NIM{Nucl. Instrum. Meth.~}
\def\PRD{Phys. Rev.~}
\def\JHEP{JHEP~}
\def\PROC{Conf. Proc.~}
\def\CPC{Comp. Phys. Commun.~}


%%%%%%%%%%%%%%%%%%%%%%%%%%%%%%%%%%%%%%% title page %%%%%%%%%%%%%%%%%%%%%%%%%%%%%%%%%%%%%%%%
\begin{titlepage}

\noindent
\begin{flushleft}
%{\tt Autumn 2019}                  \\
\end{flushleft}

\noindent
Date:      \ \ \ August 2019      \\
Version:   0.0 \\
Editors:   D.\ Britzger (britzger@mpp.mpg.de), H. Spiesberger (spiesber@uni-mainz.de)
\noindent

\vspace{2cm}
\begin{center}
\begin{Large}
{\bf Prospects for a determination of electroweak parameters with LHeC inclusive DIS data}
\end{Large}
\end{center}

\vspace{2cm}

\begin{abstract}
\noindent
%
Electroweak parameters are determined from LHeC pseudo data.
%
\end{abstract}


\vspace{6cm}

\begin{center} To be submitted to a journal \end{center}

\end{titlepage}
\sloppy

\clearpage
%
%   REMOVE the table of contents 
%\clearpage
%\tableofcontents
%\clearpage

\section*{Todo}
\begin{itemize}
\item Redo all fits with 'new' pseudo data
\item Decide, whether to  include \mW\ from real HERA data or not.
\item discuss on how to deal with 50\,\GeV vs.\ 60\,\GeV electrons
\item finalise paper
\end{itemize}
\clearpage



%%%%%%%%%%%%%%%%%%%%%%%%%%%%%%%%%%%%%%% body starts here %%%%%%%%%%%%%%%%%%%%%%%%%%%%%%%%%%%%%%%%

%-----------------------------------------------------------------------
%   Introduction
%-----------------------------------------------------------------------
\section{Introduction}
The EW parameters are determined together with the parameters of
parton density functions (PDFs) of the proton in combined fits, thus
accounting for their correlated uncertainties.

%

\section{Inclusive NC and CC DIS and generation of LHeC pseudo-data}
{\color{red} This is from the H1 paper.}
NC interactions in the process $e^\pm p\rightarrow e^\pm X$ are
mediated by a virtual photon $(\gamma)$ or $Z$ boson in the
$t$-channel, and the cross section is expressed in terms of
generalised structure functions $\tilde{F}_2^\pm$, $x\tilde{F}_3^\pm$
and $\tilde{F}_{\rm L}^\pm$ at EW leading order (LO)
as
\begin{equation}
  \frac{d^2\sigma^{\rm NC}(e^\pm p)}{dxd\Qsq} = \frac{2\pi\alpha^2}{xQ^4}\left[Y_+\tilde{F}_2^\pm(x,\Qsq) \mp Y_{-}  x\tilde{F}_3^\pm(x,\Qsq) - y^2 \tilde{F}_{\rm L}^\pm(x,\Qsq)\right]~,
  \label{eq:cs}
\end{equation}
where $\alpha$ is the fine structure constant and $x$ denotes the
Bjorken scaling variable (see e.g.~\cite{CooperSarkar:1998ug}).
The helicity dependence of the interaction is contained in the terms $Y_\pm = 1\pm(1-y)^2$ with $y$ being the inelasticity of the process.
The generalised structure functions can be separated into contributions
from pure $\gamma$- and $Z$-exchange and their interference~\cite{Klein:1983vs},
\begin{align}
  \tilde{F}_2^\pm
  &= F_2
  -(\ve\pm P_e\gae)\varkappa_ZF_2^{\gamma Z}
  +\left[(\ve\ve+\gae\gae)\pm2P_e\ve\gae\right]\varkappa_Z^2F_2^Z~,
  \\
  \tilde{F}_3^\pm
  &=~~
  -(\gae\pm P_e\ve)\varkappa_ZF_3^{\gamma Z}
  +\left[2\ve\gae\pm P_e(\ve\ve+\gae\gae)\right]\varkappa_Z^2F_3^Z~,
\end{align}
and similarly for $\tilde{F}_L$. The variables $g^e_V$ and $g^e_A$
stand for the vector and axial-vector couplings of the lepton $e^\pm$
to the $Z$ boson.





%-----------------------------------------------------------------------
%   Results from HERA combined data
%-----------------------------------------------------------------------
\boldmath\section{Determination of the $W$-boson mass from HERA combined data}


%-----------------------------------------------------------------------
%   Results from LHeC inclusive DIS data
%-----------------------------------------------------------------------
\section{Prospects for LHeC}

\subsection{Mass determinations}\label{sec:mass}


\subsection{Weak neutral-current couplings }\label{sec:couplings}

\subsection{The $\rhop{}$, $\kapp{}$ and $\rhopW{}$ parameters}\label{sec:rho}


\subsection{The effective weak mixing angle}\label{sec:sin2thw}






%-----------------------------------------------------------------------
%                            Summary
%-----------------------------------------------------------------------
\section{Summary}
\label{sect:Conclusion}



%%%%%%%%%%%%%%%%%%%%%%%%%%%%%%%%%%%%%%%%%%%%%%%%%%%%%%%%%%%%%%%%%%%%%

\section*{Acknowledgements}
Acknowledgements: Z.~Zhang, A.~Sch\"oning, M.~Klein, S.~Schmitt


%%%%%%%%%%%%%%%%%%%%%%%%%%%%%%%%%%%%%%%%%%%%%%%%%%%%%%%%%%%%%%%

%%%%%%%%%%%%%%%%%%%%%%%%%%%%%%%%%%%%%%%%%%%%%%%%%%%%%%%%%%%%%%%%%



%%======================= References ==========================%%
%%%%%%%%%%%%%%%%%%%%%%%%%%%%%%%%%%%%%%%%%%%%%%%%%%%%%%%%%%%%%%%%%

\clearpage

\begin{flushleft}
\bibliography{lhec_ew}
\end{flushleft}


\end{document}
