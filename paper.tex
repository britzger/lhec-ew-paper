% ================================================================
% LaTeX file with prefered layout for paper drafts
% use: dvips -D600 file-name
% ================================================================
%%%%%%%%%%%%% LATEX HEADER
%%%%%%%%%%%%% DO NOT DELET NOT CHANGE ANYTHING IN THE HEADER
%%%%%%%%%%%%% UNLESS CLEARLY RECOMMENDED

\documentclass[12pt]{article}
%\usepackage[T1]{fontenc}
\usepackage[utf8]{inputenc}
%\bibliographystyle{desy19-080}
\bibliographystyle{utphys}
 %Choose a bibliograhpic style
\usepackage{amsmath}
\usepackage{amssymb}
%\usepackage{times}
%\usepackage{txfonts}
\usepackage[mathlines,displaymath]{lineno}
\DeclareMathAlphabet{\mathbold}{OML}{txr}{b}{it}
%\usepackage[font={small,it}]{caption}
\usepackage[font={small}]{caption}

\usepackage{multirow}
%\usepackage{units}
%\usepackage{hhline}
\usepackage{dsfont}
%\usepackage{subcaption}
%\usepackage{pdflscape}
\usepackage{graphicx}
\usepackage{rotating}
\usepackage{paralist} % compactitem
\graphicspath{{figures/}{./}}

%\usepackage{longtable}
\usepackage{xspace}
%\usepackage{bm} % 'bold' math symbols (better use \usepackage{newtxtext,newtxmath}, if available on latex distribution)
\usepackage{acronym}
\usepackage{dcolumn}
\newcolumntype{L}{D{.}{.}{2,5}}
%\usepackage{bbold}
\usepackage{booktabs} % for nice tables

\usepackage{cite} % to allow a range of citations

%%%%%%%%%%%%% Comment the next two lines to remove the line numbering
\usepackage[]{lineno}
\linenumbers
\usepackage{soul}
\usepackage{xcolor}
\newcommand{\TODO}{\color{red}TODO\xspace}
%%%%%%%%%%%%%%

\newcommand{\includegraphicss}[2][]{\fbox{\includegraphics[#1]{#2}}}
%\newcommand{\includegraphicss}[2][]{\includegraphics[#1]{#2}}



\usepackage{hyperref} % has to be last package loaded
\hypersetup{colorlinks=true, urlcolor=blue}
%\usepackage{cite} % Use this package to display references as [21-25], which in contrast removes the hyperlink
\usepackage{cite} % DB: enable also clickable references (must be loaded after hyperref)
\hypersetup{
  colorlinks,
  citecolor=blue,
  linkcolor=red,
  urlcolor=blue
  }
  
%%%%%%%%%%%%
\renewcommand{\topfraction}{1.0}
\renewcommand{\bottomfraction}{1.0}
\renewcommand{\textfraction}{0.0}
\renewcommand{\arraystretch}{1.05} % make lines a bit larger for tables
\newlength{\dinwidth}
\newlength{\dinmargin}
\setlength{\dinwidth}{21.0cm}
\textheight22.0cm \textwidth16.0cm
\setlength{\dinmargin}{\dinwidth}
\setlength{\unitlength}{1mm}
\addtolength{\dinmargin}{-\textwidth}
\setlength{\dinmargin}{0.5\dinmargin}
\oddsidemargin -1.0in
\addtolength{\oddsidemargin}{\dinmargin}
\setlength{\evensidemargin}{\oddsidemargin}
\setlength{\marginparwidth}{0.9\dinmargin}
\marginparsep 8pt \marginparpush 5pt
\topmargin -42pt
\headheight 12pt
\headsep 30pt \footskip 44pt
\parskip 3mm plus 2mm minus 2mm

% do not indent first line of paragraph!
%\setlength{\parindent}{0pt}
\usepackage{parskip}
\hyphenation{pa-ra-me-ters}


%%%%%%%%%%%%%%%% END OF LATEX HEADER
%===============================title page=============================
\begin{document}  

%Roman numbers
\makeatletter
\newcommand*{\rom}[1]{\expandafter\@slowromancap\romannumeral #1@}
\makeatother

%% uncertainties: Absolute and relative uncertainties
\newcommand{\delrel}{\ensuremath{\delta}}
\newcommand{\delabs}{\ensuremath{\Delta}}
\newcommand{\dabs}[2][1]{\ensuremath{\delabs^{{\rm{#1}}}_{{#2}}}}
\newcommand{\drel}[2][1]{\ensuremath{\delrel^{{\rm{#1}}}_{{#2}}}}

\newcommand{\muf}{\ensuremath{\mu_{f}}\xspace}
\newcommand{\mur}{\ensuremath{\mu_{r}}\xspace}
\newcommand{\as}{\ensuremath{\alpha_s}\xspace}
\newcommand{\asmz}{\ensuremath{\alpha_s(M_Z)}\xspace}
\newcommand{\asmur}{\ensuremath{\alpha_s(\mur)}\xspace}
\newcommand{\aem}{\ensuremath{\alpha_{\mathrm{em}}}\xspace}
\newcommand{\Lumi}{\ensuremath{\mathcal{L}}}
\newcommand{\pb}{\rm pb}
\newcommand{\invpb}{\ensuremath{\rm{pb}^{-1}}}
\newcommand{\PDF}{\ensuremath{{\rm PDF}}\xspace}

\newcommand{\ex}{\ensuremath{{\rm exp}}\xspace}
\newcommand{\pdf}{\ensuremath{{\rm pdf}}\xspace}

%% observables
\newcommand{\xbj}{\ensuremath{x}\xspace}
\newcommand{\Qsq}{\ensuremath{Q^2}\xspace}

%% ep scattering
\newcommand{\ep}{\ensuremath{e^+}\xspace}
\newcommand{\emm}{\ensuremath{e^-}\xspace}
\newcommand{\epm}{\ensuremath{e^\pm}\xspace}

%% tex
\newcommand{\etal}{{\it et al.}}
\newcommand{\eq}{equation}
\newcommand{\fig}{figure}
\newcommand{\tab}{table}

%% units
\newcommand{\MeV}{\ensuremath{\mathrm{MeV}}\xspace}
\newcommand{\GeV}{\ensuremath{\mathrm{GeV}}\xspace}
\newcommand{\TeV}{\ensuremath{\mathrm{TeV}}\xspace}
\newcommand{\GeVsq}{\ensuremath{\mathrm{GeV}^2}\xspace}
\newcommand{\Ord}{\ensuremath{\mathcal{O}}}

%% EW parameters
\newcommand{\sw}{\ensuremath{\sin^2\hspace*{-0.15em}\theta_W}}
\newcommand{\sweff}{\ensuremath{\sin^2\hspace*{-0.15em}\theta_{\textrm{W},f}^\textrm{eff}}} % (\mz^2)
\newcommand{\sweffl}{\ensuremath{\sin^2\hspace*{-0.15em}\theta_{\textrm{W},\ell}^\textrm{eff}}} % (\mz^2)
\newcommand{\dr}{\ensuremath{\Delta r}}
\newcommand{\gf}{\ensuremath{G_{\rm F}}}
%%%%%%%%%%%%%%%%%%%%%%%%%%%%%%%%%%%%%%%%%%%%%%%%%%%%%%%%%%%%

\newcommand{\rhopW}[2][]{\ensuremath{\rho^{\prime}_{\text{CC}#2}}}
\newcommand{\rhop}[2][] {\ensuremath{\rho^{\prime}_{\text{NC}#2}}}
%\newcommand{\kapp}[2][] {\ensuremath{\kappa^{\prime}_{\text{NC}#2}}}
\newcommand{\kapp}[2][] {\ensuremath{\kappa^{\prime}_{#2}}}
\newcommand{\rhopu}{\rhop{, u}}
\newcommand{\kappu}{\kapp{, u}}
\newcommand{\rhopd}{\rhop{, d}}
\newcommand{\kappd}{\kapp{, d}}
\newcommand{\rhope}{\rhop{, e}}
\newcommand{\kappe}{\kapp{, e}}
\newcommand{\rhopq}{\rhop{, q}}
\newcommand{\kappq}{\kapp{, q}}
\newcommand{\rhopf}{\rhop{, f}}
\newcommand{\kappf}{\kapp{, f}}
%\newcommand{\kapz}{\ensuremath{\kappa_{\text{NC}, f}}}
\newcommand{\kapz}{\ensuremath{\kappa_{f}}}
\newcommand{\rhoz}{\ensuremath{\rho_{\text{NC},f}}}
\newcommand{\Itf}{\ensuremath{I^3_{{\rm L},f}}}
\newcommand{\Itq}{\ensuremath{I^3_{{\rm L},q}}}


\newcommand{\chisq}{\ensuremath{\chi^{2}}}

% couplings
\newcommand{\ad} {\ensuremath{g_A^d}}
\newcommand{\vd} {\ensuremath{g_V^d}}
\newcommand{\au} {\ensuremath{g_A^u}}
\newcommand{\vu} {\ensuremath{g_V^u}}
\newcommand{\aq} {\ensuremath{g_A^q}}
\newcommand{\vq} {\ensuremath{g_V^q}}
\newcommand{\gae}{\ensuremath{g_A^e}}
\newcommand{\ve} {\ensuremath{g_V^e}}
\newcommand{\af} {\ensuremath{g_A^f}}
\newcommand{\vf} {\ensuremath{g_V^f}}

%% masses
\newcommand{\mH} {\ensuremath{m_H}}
\newcommand{\mw}{\ensuremath{m_W}}
\newcommand{\mW}{\mw}
\newcommand{\mz}{\ensuremath{m_Z}}
\newcommand{\mZ}{\mz}
\newcommand{\mt}{\ensuremath{m_t}}
\newcommand{\mWprop} {\ensuremath{m^{\rm prop}_W}}
\newcommand{\mWGfW} {\ensuremath{m^{(\gf,\mW)}_W}}

% Journal macro
\def\Journal#1#2#3#4{{#1}~{\bf #2} (#3) #4}
%
\def\NPB{Nucl. Phys.~}
\def\PRL{Phys. Rev. Lett.~}
\def\EPJC{Eur. Phys. J.~}
\def\PLB{Phys. Lett.~}
\def\NIM{Nucl. Instrum. Meth.~}
\def\PRD{Phys. Rev.~}
\def\JHEP{JHEP~}
\def\PROC{Conf. Proc.~}
\def\CPC{Comp. Phys. Commun.~}


%%%%%%%%%%%%%%%%%%%%%%%%%%%%%%%%%%%%%%% title page %%%%%%%%%%%%%%%%%%%%%%%%%%%%%%%%%%%%%%%%
\begin{titlepage}

\noindent
\begin{flushleft}
%{\tt Autumn 2019}                  \\
\end{flushleft}

\noindent
\begin{flushright}
  MITP/20-038\\
  MPP-2020-110 
\end{flushright}

%\noindent
%Date:      \ \ \ July 6, 2020      \\
%Version:   2.2 \\
%Editors:   D.\ Britzger (britzger@mpp.mpg.de), H. Spiesberger (spiesber@uni-mainz.de)
%\noindent


\vspace{1.0cm}

\begin{center}
\begin{Large}
{\bf 
Electroweak Physics in Inclusive Deep Inelastic 
\\[1ex]
Scattering at the LHeC}
\end{Large}
\end{center}

\vspace{1.0cm}


\begin{center}
  Daniel Britzger\,$^1$, Max Klein\,$^{2}$ and Hubert Spiesberger\,$^{3}$ \\
\vspace{0.5cm}
\small
\it
$^1$~~Max-Planck-Institut f{\"u}r Physik, F{\"o}hringer Ring 6, D-80805 M{\"u}nchen, Germany \\
$^2$~~University of Liverpool, Oxford Street, UK-L69 7ZE Liverpool, United Kingdom \\
$^3$~~PRISMA$^+$ Cluster of Excellence, Institut f{\"u}r Physik, Johannes-Gutenberg-Universit{\"a}t, 
Staudinger Weg 7, D-55099 Mainz, Germany 
\end{center}

\vspace{1.5cm}

\begin{abstract}
\noindent
The proposed electron-proton collider LHeC is a unique facility 
where electroweak interactions can be studied with a very high 
precision in a largely unexplored kinematic regime of 
spacelike momentum transfer. 
%
We have simulated inclusive neutral- and charged-current 
deep-inelastic lepton proton scattering cross section data at 
center-of-mass energies of 1.2 and 1.3\,\TeV.
% 
Based on simultaneous fits of electroweak physics parameters and 
parton distribution functions, we estimate the uncertainties of 
Standard Model parameters as well as a number of parameters 
describing physics beyond the Standard Model, for instance 
the oblique parameters $S$, $T$, and $U$. 
%
An unprecedented precision at the sub-percent level is expected 
for the measurement of the weak neutral-current couplings of the 
light-quarks to the $Z$ boson, $g_{A/V}^{u/d}$, improving their 
present precision by more than an order of magnitude.
%
The weak mixing angle can be determined with a precision of 
about $\Delta \sw = \pm 0.00015$, and its scale dependence
can be studied in the energy range between about $25$ and 
$700\,\GeV$.
%
An indirect determination of the $W$-boson mass in the on-shell 
scheme is possible with an experimental uncertainty down to 
$\Delta\mW=\pm6\,\MeV$. 
%
We discuss how the uncertainties of such measurements in 
deep-inelastic scattering compare with those from measurements 
in the timelike domain, e.g.\ at the $Z$-pole, and which aspects 
of the electroweak interaction are unique to measurements at the 
LHeC, for instance electroweak parameters in charged-current 
interactions.
%
We conclude that the LHeC will determine electroweak 
physics parameters, in the spacelike region, with unprecedented
precision leading to thorough tests of the Standard Model
and possibly beyond.
%We conclude, that LHeC data will complement, with a high 
%experimental precision, the determination of electroweak physics 
%parameters from other past or ongoing experiments.

%
\end{abstract}

\end{titlepage}
\sloppy

\clearpage
%
%   REMOVE the table of contents 
%\clearpage
%\tableofcontents
%\clearpage


%%%%%%%%%%%%%%%%%%%%%%%%%%%%%%%%%%%%%%% body starts here %%%%%%%%%%%%%%%%%%%%%%%%%%%%%%%%%%%%%%%%
% ================================================================
% LaTeX file with prefered layout for H1 paper drafts
% use: dvips -D600 file-name
% ================================================================
%%%%%%%%%%%%% LATEX HEADER
%%%%%%%%%%%%% DO NOT DELET NOT CHANGE ANYTHING IN THE HEADER
%%%%%%%%%%%%% UNLESS CLEARLY RECOMMENDED

\documentclass[12pt]{article}
%\usepackage[T1]{fontenc}
\usepackage[utf8]{inputenc}
%\bibliographystyle{desy19-080}
\bibliographystyle{utphys}
 %Choose a bibliograhpic style
\usepackage{amsmath}
\usepackage{amssymb}
%\usepackage{times}
\usepackage{txfonts}
\usepackage[mathlines,displaymath]{lineno}
\DeclareMathAlphabet{\mathbold}{OML}{txr}{b}{it}
%\usepackage[font={small,it}]{caption}
\usepackage[font={small}]{caption}

\usepackage{multirow}
%\usepackage{units}
%\usepackage{hhline}
\usepackage{dsfont}
\usepackage{xcolor}
%\usepackage{subcaption}
%\usepackage{pdflscape}
\usepackage{graphicx}
\usepackage{rotating}
\usepackage{paralist} % compactitem

%\usepackage{longtable}
\usepackage{xspace}
%\usepackage{bm} % 'bold' math symbols (better use \usepackage{newtxtext,newtxmath}, if available on latex distribution)
\usepackage{acronym}
\usepackage{dcolumn}
\newcolumntype{L}{D{.}{.}{2,5}}
%\usepackage{bbold}

%%%%%%%%%%%%% Comment the next two lines to remove the line numbering
%\usepackage[]{lineno}
%\linenumbers
%%%%%%%%%%%%%%

\newcommand{\includegraphicss}[2][]{\fbox{\includegraphics[#1]{#2}}}
%\newcommand{\includegraphicss}[2][]{\includegraphics[#1]{#2}}



\usepackage{hyperref} % has to be last package loaded
\hypersetup{colorlinks=true, urlcolor=blue}
%\usepackage{cite} % Use this package to display references as [21-25], which in contrast removes the hyperlink
\usepackage{cite} % DB: enable also clickable references (must be loaded after hyperref)
\hypersetup{
  colorlinks,
  citecolor=blue,
  linkcolor=red,
  urlcolor=blue
  }
  
%%%%%%%%%%%%
\renewcommand{\topfraction}{1.0}
\renewcommand{\bottomfraction}{1.0}
\renewcommand{\textfraction}{0.0}
\renewcommand{\arraystretch}{1.25} % make lines a bit larger for tables
\newlength{\dinwidth}
\newlength{\dinmargin}
\setlength{\dinwidth}{21.0cm}
\textheight23.5cm \textwidth16.0cm
\setlength{\dinmargin}{\dinwidth}
\setlength{\unitlength}{1mm}
\addtolength{\dinmargin}{-\textwidth}
\setlength{\dinmargin}{0.5\dinmargin}
\oddsidemargin -1.0in
\addtolength{\oddsidemargin}{\dinmargin}
\setlength{\evensidemargin}{\oddsidemargin}
\setlength{\marginparwidth}{0.9\dinmargin}
\marginparsep 8pt \marginparpush 5pt
\topmargin -42pt
\headheight 12pt
\headsep 30pt \footskip 24pt
\parskip 3mm plus 2mm minus 2mm

% do not indent first line of paragraph!
%\setlength{\parindent}{0pt}
\usepackage{parskip}
\hyphenation{pa-ra-me-ters}


%%%%%%%%%%%%%%%% END OF LATEX HEADER
%===============================title page=============================
\begin{document}  

%Roman numbers
\makeatletter
\newcommand*{\rom}[1]{\expandafter\@slowromancap\romannumeral #1@}
\makeatother

%% uncertainties: Absolute and relative uncertainties
\newcommand{\delrel}{\ensuremath{\delta}}
\newcommand{\delabs}{\ensuremath{\Delta}}
\newcommand{\dabs}[2][1]{\ensuremath{\delabs^{{\rm{#1}}}_{{#2}}}}
\newcommand{\drel}[2][1]{\ensuremath{\delrel^{{\rm{#1}}}_{{#2}}}}
\newcommand{\TODO}{\color{red}TODO\xspace}

\newcommand{\muf}{\ensuremath{\mu_{f}}\xspace}
\newcommand{\mur}{\ensuremath{\mu_{r}}\xspace}
\newcommand{\as}{\ensuremath{\alpha_s}\xspace}
\newcommand{\asmz}{\ensuremath{\alpha_s(M_Z)}\xspace}
\newcommand{\asmur}{\ensuremath{\alpha_s(\mur)}\xspace}
\newcommand{\aem}{\ensuremath{\alpha_{\mathrm{em}}}\xspace}
\newcommand{\Lumi}{\ensuremath{\mathcal{L}}}
\newcommand{\pb}{\rm pb}
\newcommand{\invpb}{\ensuremath{\rm{pb}^{-1}}}
\newcommand{\PDF}{\ensuremath{{\rm PDF}}\xspace}

\newcommand{\ex}{\ensuremath{{\rm exp}}\xspace}
\newcommand{\pdf}{\ensuremath{{\rm pdf}}\xspace}

%% observables
\newcommand{\xbj}{\ensuremath{x}\xspace}
\newcommand{\Qsq}{\ensuremath{Q^2}\xspace}

%% ep scattering
\newcommand{\ep}{\ensuremath{e^+}\xspace}
\newcommand{\emm}{\ensuremath{e^-}\xspace}
\newcommand{\epm}{\ensuremath{e^\pm}\xspace}

%% tex
\newcommand{\etal}{{\it et al.}}
\newcommand{\eq}{equation}
\newcommand{\fig}{figure}
\newcommand{\tab}{table}

%% units
\newcommand{\MeV}{\ensuremath{\mathrm{MeV}}\xspace}
\newcommand{\GeV}{\ensuremath{\mathrm{GeV}}\xspace}
\newcommand{\GeVsq}{\ensuremath{\mathrm{GeV}^2}\xspace}
\newcommand{\Ord}{\ensuremath{\mathcal{O}}}

%% EW parameters
\newcommand{\sw}{\ensuremath{{\rm sin}^2\theta_W}}
\newcommand{\dr}{\ensuremath{\Delta r}}
\newcommand{\gf}{\ensuremath{G_{\rm F}}}

\newcommand{\rhopW}[2][]{\ensuremath{\rho^{\prime}_{\text{CC}#2}}}
\newcommand{\rhop}[2][] {\ensuremath{\rho^{\prime}_{\text{NC}#2}}}
\newcommand{\kapp}[2][] {\ensuremath{\kappa^{\prime}_{\text{NC}#2}}}
\newcommand{\rhopu}{\rhop{, u}}
\newcommand{\kappu}{\kapp{, u}}
\newcommand{\rhopd}{\rhop{, d}}
\newcommand{\kappd}{\kapp{, d}}
\newcommand{\rhope}{\rhop{, e}}
\newcommand{\kappe}{\kapp{, e}}
\newcommand{\rhopq}{\rhop{, q}}
\newcommand{\kappq}{\kapp{, q}}
\newcommand{\rhopf}{\rhop{, f}}
\newcommand{\kappf}{\kapp{, f}}
\newcommand{\kapz}{\ensuremath{\kappa_{\text{NC}, f}}}
\newcommand{\rhoz}{\ensuremath{\rho_{\text{NC},f}}}
\newcommand{\Itf}{\ensuremath{I^3_{{\rm L},f}}}
\newcommand{\Itq}{\ensuremath{I^3_{{\rm L},q}}}


\newcommand{\chisq}{\ensuremath{\chi^{2}}}

% couplings
\newcommand{\ad} {\ensuremath{g_A^d}}
\newcommand{\vd} {\ensuremath{g_V^d}}
\newcommand{\au} {\ensuremath{g_A^u}}
\newcommand{\vu} {\ensuremath{g_V^u}}
\newcommand{\aq} {\ensuremath{g_A^q}}
\newcommand{\vq} {\ensuremath{g_V^q}}
\newcommand{\gae}{\ensuremath{g_A^e}}
\newcommand{\ve} {\ensuremath{g_V^e}}


%% masses
\newcommand{\mt} {\ensuremath{m_t}}
\newcommand{\mW} {\ensuremath{m_W}}
\newcommand{\mWprop} {\ensuremath{m^{\rm prop}_W}}
\newcommand{\mWGfW} {\ensuremath{m^{(\gf,\mW)}_W}}
\newcommand{\mZ} {\ensuremath{m_Z}}
\newcommand{\mH} {\ensuremath{m_H}}


% Journal macro
\def\Journal#1#2#3#4{{#1}~{\bf #2} (#3) #4}
%
\def\NPB{Nucl. Phys.~}
\def\PRL{Phys. Rev. Lett.~}
\def\EPJC{Eur. Phys. J.~}
\def\PLB{Phys. Lett.~}
\def\NIM{Nucl. Instrum. Meth.~}
\def\PRD{Phys. Rev.~}
\def\JHEP{JHEP~}
\def\PROC{Conf. Proc.~}
\def\CPC{Comp. Phys. Commun.~}


%%%%%%%%%%%%%%%%%%%%%%%%%%%%%%%%%%%%%%% title page %%%%%%%%%%%%%%%%%%%%%%%%%%%%%%%%%%%%%%%%
\begin{titlepage}

\noindent
\begin{flushleft}
%{\tt Autumn 2019}                  \\
\end{flushleft}

\noindent
Date:      \ \ \ August 2019      \\
Version:   0.0 \\
Editors:   D.\ Britzger (britzger@mpp.mpg.de), H. Spiesberger (spiesber@uni-mainz.de)
\noindent

\vspace{2cm}
\begin{center}
\begin{Large}
{\bf Prospects for a determination of electroweak parameters with LHeC inclusive DIS data}
\end{Large}
\end{center}

\vspace{2cm}

\begin{abstract}
\noindent
%
Electroweak parameters are determined from LHeC pseudo data.
%
Test.
%
\end{abstract}


\vspace{6cm}

\begin{center} To be submitted to a journal \end{center}

\end{titlepage}
\sloppy

\clearpage
%
%   REMOVE the table of contents 
%\clearpage
%\tableofcontents
%\clearpage

\section*{Todo}
\begin{itemize}
\item Redo all fits with 'new' pseudo data
\item Decide, whether to  include \mW\ from real HERA data or not.
\item discuss on how to deal with 50\,\GeV vs.\ 60\,\GeV electrons
\item finalise paper
\end{itemize}
\clearpage



%%%%%%%%%%%%%%%%%%%%%%%%%%%%%%%%%%%%%%% body starts here %%%%%%%%%%%%%%%%%%%%%%%%%%%%%%%%%%%%%%%%

%-----------------------------------------------------------------------
%   Introduction
%-----------------------------------------------------------------------
\section{Introduction}
The EW parameters are determined together with the parameters of
parton density functions (PDFs) of the proton in combined fits, thus
accounting for their correlated uncertainties.

%

\section{Inclusive NC and CC DIS and generation of LHeC pseudo-data}
{\color{red} This is from the H1 paper.}
NC interactions in the process $e^\pm p\rightarrow e^\pm X$ are
mediated by a virtual photon $(\gamma)$ or $Z$ boson in the
$t$-channel, and the cross section is expressed in terms of
generalised structure functions $\tilde{F}_2^\pm$, $x\tilde{F}_3^\pm$
and $\tilde{F}_{\rm L}^\pm$ at EW leading order (LO)
as
\begin{equation}
  \frac{d^2\sigma^{\rm NC}(e^\pm p)}{dxd\Qsq} = \frac{2\pi\alpha^2}{xQ^4}\left[Y_+\tilde{F}_2^\pm(x,\Qsq) \mp Y_{-}  x\tilde{F}_3^\pm(x,\Qsq) - y^2 \tilde{F}_{\rm L}^\pm(x,\Qsq)\right]~,
  \label{eq:cs}
\end{equation}
where $\alpha$ is the fine structure constant and $x$ denotes the
Bjorken scaling variable (see e.g.~\cite{CooperSarkar:1998ug}).
The helicity dependence of the interaction is contained in the terms $Y_\pm = 1\pm(1-y)^2$ with $y$ being the inelasticity of the process.
The generalised structure functions can be separated into contributions
from pure $\gamma$- and $Z$-exchange and their interference~\cite{Klein:1983vs},
\begin{align}
  \tilde{F}_2^\pm
  &= F_2
  -(\ve\pm P_e\gae)\varkappa_ZF_2^{\gamma Z}
  +\left[(\ve\ve+\gae\gae)\pm2P_e\ve\gae\right]\varkappa_Z^2F_2^Z~,
  \\
  \tilde{F}_3^\pm
  &=~~
  -(\gae\pm P_e\ve)\varkappa_ZF_3^{\gamma Z}
  +\left[2\ve\gae\pm P_e(\ve\ve+\gae\gae)\right]\varkappa_Z^2F_3^Z~,
\end{align}
and similarly for $\tilde{F}_L$. The variables $g^e_V$ and $g^e_A$
stand for the vector and axial-vector couplings of the lepton $e^\pm$
to the $Z$ boson.





%-----------------------------------------------------------------------
%   Results from HERA combined data
%-----------------------------------------------------------------------
\section{Determination of the $W$-boson mass from HERA combined data}


%-----------------------------------------------------------------------
%   Results from LHeC inclusive DIS data
%-----------------------------------------------------------------------
\section{Prospects for LHeC}

\subsection{Mass determinations}\label{sec:mass}


\subsection{Weak neutral-current couplings }\label{sec:couplings}

\subsection{The $\rhop{}$, $\kapp{}$ and $\rhopW{}$ parameters}\label{sec:rho}

\section{Determinations of single parameters}
Maximum sensitivity. Reasonable to study deviations from SM with highest sensitivity.
\begin{table}[h]
  \footnotesize
  \centering
  \begin{tabular}{lcr@{$\,\pm\,$}lr@{$\,\pm\,$}l}
    \hline
    Fit parameters & Parameter & \multicolumn{2}{c}{LHeC}& \multicolumn{2}{c}{FCC}  \\
    \hline
    \rhopu+PDF & \rhopu &  1  &  0.009    & 1  & 0.004  \\
    \kappu+PDF & \kappu &  1  &  0.004    & 1  & 0.003  \\
    \rhopd+PDF & \rhopd &  1  &  0.014    & 1  & 0.006  \\
    \kappd+PDF & \kappd &  1  &  0.023    & 1  & 0.013  \\
    \rhope+PDF & \rhope &  1  &  0.006    & 1  & 0.003  \\
    \kappe+PDF & \kappe &  1  &  0.003    & 1  & 0.002  \\
    \hline
    \rhopq+PDF & \rhopq &  1  &  0.0059    & 1  & 0.0027  \\
    \kappq+PDF & \kappq &  1  &  0.0038    & 1  & 0.0024  \\
    \hline
    \rhopf+PDF & \rhopf &  1  &  0.0031    & 1  & 0.0015  \\ %0.00145641
    \kappf+PDF & \kappf &  1  &  0.0019    & 1  & 0.0011  \\
    \hline
    \\
    \multicolumn{3}{l}{Expectations for \rhopW{,f} (CC)} \\
    \hline
    % log.final.PAR19eq50.3p.PDF.2.txt
    \rhopW{,f}+PDF   & \rhopW{,f}                &  1  &   0.0043  & 1  & 0.0027 \\
    \rhopW{,eq}+PDF  & \rhopW{,eq}               &  1  &   0.0027  & 1  &  0.0011\\
    \rhopW{,e\bar{q}}+PDF  & \rhopW{,e\bar{q}}   &  1  &   0.0030  & 1  &  0.0012\\
    \hline
  \end{tabular}
  \caption{
    Results for $\rhop{}$, $\kapp{}$ and \rhopW{} parameters, and their
    correlation coefficients.
  }
  \label{tab:rhopwithcorrelations}
\end{table}


\section{FCC and LHeC}
\begin{table}[h]
  \footnotesize
  \centering
  \begin{tabular}{lcr@{$\,\pm\,$}lr@{$\,\pm\,$}l}
    \hline
    Fit parameters & Parameter & \multicolumn{2}{c}{LHeC}& \multicolumn{2}{c}{FCC}  \\
    \hline
    % log.fcc.6p.u.d.e.txt
    % EPRC.rhopu                =            1   +/-   0.00864794
    % EPRC.zkapu                =            1   +/-   0.00886882
    % EPRC.rhopd                =            1   +/-   0.028591
    % EPRC.zkapd                =            1   +/-   0.0438381
    % EPRC.rhope                =            1   +/-   0.0136914
    % EPRC.zkape                =            1   +/-   0.00510635
    \rhopd+\kappd+\rhopu+\kappu+\rhop{,e}+\kapp{,e}+PDF
    & \rhopu &  1  &  0.031    & 1  & 0.009 \\
    & \kappu &  1  &  0.013    & 1  & 0.009 \\
    & \rhopd &  1  &  0.062    & 1  & 0.029   \\
    & \kappd &  1  &  0.076    & 1  & 0.044  \\
    & \rhope &  1  &  0.036    & 1  & 0.014  \\
    & \kappe &  1  &  0.008    & 1  & 0.005 \\
    \hline
    % /nfs/dust/h1/group/britzger/alpos/Alpos/../log.fcc.4p.u.d.txt
    % EPRC.rhopu                =            1   +/-   0.00748481
    % EPRC.zkapu                =            1   +/-   0.00663437
    % EPRC.rhopd                =            1   +/-   0.0159018
    % EPRC.zkapd                =            1   +/-   0.0419873
    %lhec
    % EPRC.rhopu                =            1   +/-   0.0185425
    % EPRC.zkapu                =            1   +/-   0.0102184
    % EPRC.rhopd                =            1   +/-   0.0394555
    % EPRC.zkapd                =            1   +/-   0.0757502
    \rhopd+\kappd+\rhopu+\kappu+PDF
    & \rhopu &  1  &  0.019  & 1  & 0.007   \\
    & \kappu &  1  &  0.010  & 1  & 0.007   \\
    & \rhopd &  1  &  0.039  & 1  & 0.016   \\
    & \kappd &  1  &  0.076  & 1  & 0.042   \\
    \hline
    % /nfs/dust/h1/group/britzger/alpos/Alpos/../log.fcc.4p.e.q.txt
    %rhopq                     =            1   +/-   0.00827489
    %zkapq                     =            1   +/-   0.005199
    %EPRC.rhope                =            1   +/-   0.0100664
    %EPRC.zkape                =            1   +/-   0.00504328
    \rhop{,q}+\kapp{,q}+\rhop{,e}+\kapp{,e}+PDF
    & \rhop{,q} & 1 &  0.029  & 1 &  0.008  \\
    & \kapp{,q} & 1 &  0.007  & 1 &  0.005 \\
    & \rhop{,e} & 1 &  0.032  & 1 &  0.010 \\
    & \kapp{,e} & 1 &  0.008  & 1 &  0.005 \\
    \hline
    % /nfs/dust/h1/group/britzger/alpos/Alpos/../log.fcc.4p.e.q.txt
    %rhopq                     =            1   +/-   0.00827489
    %zkapq                     =            1   +/-   0.005199
    %EPRC.rhope                =            1   +/-   0.0100664
    %EPRC.zkape                =            1   +/-   0.00504328
    \rhop{,q}+\kapp{,q}+\rhop{,e}+\kapp{,e}+PDF
    & \rhop{,q} & 1 &  0.029  & 1 &  0.008  \\
    & \kapp{,q} & 1 &  0.007  & 1 &  0.005 \\
    & \rhop{,e} & 1 &  0.032  & 1 &  0.010 \\
    & \kapp{,e} & 1 &  0.008  & 1 &  0.005 \\
    \hline
    % /nfs/dust/h1/group/britzger/alpos/Alpos/
    \rhopu+\kappu+PDF
    & \rhopu &  1  &  0.011    & 1 & 0.005\\
    & \kappu &  1  &  0.005    & 1 & 0.003\\
    \hline
    % /nfs/dust/h1/group/britzger/alpos/Alpos/
    \rhopd+\kappd+PDF
    & \rhopd &  1  &  0.022    &  1  & 0.011 \\
    & \kappd &  1  &  0.038    &  1  & 0.021 \\
    \hline
    \rhop{,e}+\kapp{,e}+PDF
    & \rhope &  1  & 0.009      & 1  & 0.005 \\
    & \kappe &  1  & 0.005      & 1  & 0.003  \\
    \hline
    \rhop{,f}+\kapp{,f}+PDF
    & \rhope &  1  & 0.0042      & 1  & 0.0022 \\
    & \kappe &  1  & 0.0026      & 1  & 0.0016  \\
    \hline
    \\
    \multicolumn{3}{l}{Fits including \rhopW{,f} (CC)} \\
    \hline
    % log.final.PAR19eq50.3p.PDF.2.txt
    \rhop{,f}$+$\kapp{,f}$+$\rhopW{,f}+PDF
    & \rhop{f}   &  1  &  0.0045   & 1  & 0.0021 \\
    & \kapp{f}   &  1  &  0.0027   & 1  & 0.0015 \\
    & \rhopW{,f} &  1  &  0.0043   & 1  & 0.0027 \\
    % & $\rhop{W,f}=1.002\pm0.008$ & \\
    \hline
  \end{tabular}
  \caption{
    Results for $\rhop{}$, $\kapp{}$ and \rhopW{} parameters, and their
    correlation coefficients.
  }
  \label{tab:rhopwithcorrelations}
\end{table}


\section{LHeC}
\begin{table}[h]
  \footnotesize
  \centering
  \begin{tabular}{lr@{$\,=\,$}c@{$\,\pm\,$}l|cccccc}
    \hline
    Fit parameters & \multicolumn{3}{c}{Result} & \multicolumn{6}{l}{Correlation} \\
    \hline
    % /nfs/dust/h1/group/britzger/alpos/Alpos/
    \rhopd+\kappd+\rhopu+\kappu+\rhop{,e}+\kapp{,e}+PDF
    & \rhopu &    &      & 1.00 \\
    & \kappu &    &      & 0.  & 1.00 \\
    & \rhopd &    &      &$    $& $   $ & 1.00 \\
    & \kappd &    &      &$    $& $   $ &  & 1.00 \\
    & \rhope &    &      &      &       &  &      &  1.00 \\
    & \kappe &    &      & 0.   &       &  &      &  & 1.00 \\
    \hline
    % /nfs/dust/h1/group/britzger/alpos/Alpos/
    \rhopd+\kappd+\rhopu+\kappu+PDF
    & \rhopu &    &      & 1.00 \\
    & \kappu &    &      & 0.  & 1.00 \\
    & \rhopd &    &      &$    $& $   $ & 1.00 \\
    & \kappd &    &      &$    $& $   $ &  & 1.00 \\
    \hline
    % /nfs/dust/h1/group/britzger/alpos/Alpos/
    \rhop{,q}+\kapp{,q}+\rhop{,e}+\kapp{,e}+PDF
    & \rhop{,e} &    &     & 1.00 \\
    & \kapp{,e} &    &     & 0.  & 1.00 \\
    & \rhop{,q} &    &     &     & 0.  & 1.00 \\
    & \kapp{,q} &    &     & 0.  & $0$ & 0.0  & 1.00 \\
    \hline
    % /nfs/dust/h1/group/britzger/alpos/Alpos/
    \rhopu+\kappu+PDF
    & \rhopu &    &      & 1.00 \\
    & \kappu &    &      & 0.  & 1.00 \\
    \hline
    % /nfs/dust/h1/group/britzger/alpos/Alpos/
    \rhopd+\kappd+PDF
    & \rhopd &    &      & 1.00 \\
    & \kappd &    &      &  & 1.00 \\
    \hline
    % /nfs/dust/h1/group/britzger/alpos/Alpos/
    \rhop{,e}+\kapp{,e}+PDF
    & \rhope &    &      &  1.00 \\
    & \kappe &    &      &  & 1.00 \\
    \hline
    \\
    \multicolumn{3}{l}{Fits including \rhopW{,f} (CC)} \\
    \hline
    % log.final.PAR19eq50.3p.PDF.2.txt
    \rhop{,f}$+$\kapp{,f}$+$\rhopW{,f}+PDF
    & \rhop{f}   &    &     & 1.00 & \\
    & \kapp{f}   &    &     & 0. & 1.00  \\
    & \rhopW{,f} &    &     & 0. & $0.$ & 1.00\\
    % & $\rhop{W,f}=1.002\pm0.008$ & \\
    \hline
  \end{tabular}
  \caption{
    Results for $\rhop{}$, $\kapp{}$ and \rhopW{} parameters, and their
    correlation coefficients.
  }
  \label{tab:rhopwithcorrelations}
\end{table}

\clearpage
\begin{table}[tbhp]
  %\footnotesize \scriptsize
  \begin{center}
    \begin{tabular}{l|ccc}
 %     \multicolumn{4}{c}{\bf Uncertainty on EW parameters from inclusive DIS data} \\
      \hline
          {\bf Parameter} & {\bf HERA} & {\bf LHeC} & {\bf FCC-eh}  \\
          \hline
          $\Delta$\mW~[{MeV}]  & $ \pm63\ex 29\pdf$ & $\pm14\ex 10\pdf $ & $ \pm9\ex 4\pdf $ \\
          $\Delta$\mZ~[{MeV}]  & $ \pm56\ex 25\pdf $ & $\pm16\ex 10\pdf $ & $ \pm16\ex 10\pdf $ \\
          $\Delta$\mt~[{GeV}]  & $\pm10\ex 5\pdf $ & $\pm2.6\ex 1.7\pdf $ & $\pm1.7\ex 0.5\pdf $  \\
          $\Delta$\mH~[{GeV}]  & $ >\mathcal{O}(100\,\GeV)$ & $\pm31\ex 22\pdf $ & $\pm20\ex 4\pdf $  \\
%          \hline
%          $\Delta\gf^{\rm MOMS}$ [$10^{-8} \GeV^{-2}$]
%          & $\pm4.7\ex 1.9\pdf$
%          & $\pm1.9\ex 1.2\pdf$
%          & $\pm2.8\ex 1.8\pdf$
%          \\
          \hline
    \end{tabular}
    \caption{Summary of electroweak parameters from HERA-II data and LHeC and FCC-ep simulated data.}
    \label{tab:datasets}
  \end{center}
\end{table}

\paragraph{Neutral current\\}
Universal higher-order corrections are be taken into account by $\Qsq$-dependent form
factors $\rho_{\text{NC}}$ and $\kappa_{\text{NC}}$.
Many extensions of the Standard Model predict modifications of the weak neutral-current couplings.
These can be described conveniently by introducing additional parameters \rhop\ and \kapp{},
which can be also considerd to be \Qsq\ dependent:
\begin{align}
  g_A^f &= \sqrt{\rho_{\text{NC}, f}\rhop{,f}} \Itf
  \label{eq:gA} \, ,
  \\
  g_V^f &= \sqrt{\rho_{\text{NC}, f}\rhop{,f}} \left( \Itf - 2 Q_f \kappa_{\text{NC}, f}\kapp{,f}\sw \right)
  \label{eq:gV} \,.
\end{align}
The estimated relative uncertainties of the \rhop\ or \kapp{} achieved with the LHeC or FCC-eh data,
can also be interpreted as the relative uncertainty of a direct determination of the $\rho_{\rm NC}$ parameters
or $\sin^2\theta_w^{\rm eff}$.


\paragraph{Charged current\\}
Higher-order EW corrections to the CC cross sections are collected in form factors
$\rho_{\text{CC}, eq/e\bar{q}}$.
Similarly as for NC, modifications of the SM formalism can be expressed by introducing
the additional \rhopW{} parameters:
\begin{align}
  W_2^- &=
  x \left( (\rho_{\text{CC}, eq}\rhopW{,eq})^2 U + (\rho_{\text{CC},e\bar{q}}\rhopW{,e\bar{q}})^2 \overline{D} \right)
  \, ,
  \\
  xW_3^- &=
  x \left( (\rho_{\text{CC},eq}\rhopW{,eq})^2 U - (\rho_{\text{CC},e\bar{q}}\rhopW{,e\bar{q}})^2 \overline{D} \right)
  \, ,
\\
  W_2^+ &=
  x \left( (\rho_{\text{CC},eq}\rhopW{,eq})^2 \overline{U}+ \rho_{\text{CC},e\bar{q}}\rhopW{,e\bar{q}})^2 D \right)
  \, ,
  \\
  xW_3^+ &=
  x \left( (\rho_{\text{CC},e\bar{q}}\rhopW{,e\bar{q}})^2 D - \rho_{\text{CC},eq}\rhopW{,eq})^2 \overline{U} \right)
  \, .
\end{align}



\subsection{The effective weak mixing angle}\label{sec:sin2thw}




%-----------------------------------------------------------------------
%   Theory
%-----------------------------------------------------------------------
\clearpage
\section{Theoretical framework}
NC interactions in the process $e^\pm p\rightarrow e^\pm X$ are
mediated by a virtual photon $(\gamma)$ or $Z$ boson in the
$t$-channel, and the cross section is expressed in terms of
generalised structure functions $\tilde{F}_2^\pm$, $x\tilde{F}_3^\pm$
and $\tilde{F}_{\rm L}^\pm$ at EW leading order (LO)
as
\begin{equation}
  \frac{d^2\sigma^{\rm NC}(e^\pm p)}{dxd\Qsq} = \frac{2\pi\alpha^2}{xQ^4}\left[Y_+\tilde{F}_2^\pm(x,\Qsq) \mp Y_{-}  x\tilde{F}_3^\pm(x,\Qsq) - y^2 \tilde{F}_{\rm L}^\pm(x,\Qsq)\right]~,
  \label{eq:cs}
\end{equation}
where $\alpha$ is the fine structure constant and $x$ denotes the
Bjorken scaling variable (see e.g.~\cite{CooperSarkar:1998ug}).
The helicity dependence of the interaction is contained in the terms $Y_\pm = 1\pm(1-y)^2$ with $y$ being the inelasticity of the process.
The generalised structure functions can be separated into contributions
from pure $\gamma$- and $Z$-exchange and their interference~\cite{Klein:1983vs},
\begin{align}
  \tilde{F}_2^\pm
  &= F_2
  -(\ve\pm P_e\gae)\varkappa_ZF_2^{\gamma Z}
  +\left[(\ve\ve+\gae\gae)\pm2P_e\ve\gae\right]\varkappa_Z^2F_2^Z~,
  \\
  \tilde{F}_3^\pm
  &=~~
  -(\gae\pm P_e\ve)\varkappa_ZF_3^{\gamma Z}
  +\left[2\ve\gae\pm P_e(\ve\ve+\gae\gae)\right]\varkappa_Z^2F_3^Z~,
\end{align}
and similarly for $\tilde{F}_L$. The variables $g^e_V$ and $g^e_A$
stand for the vector and axial-vector couplings of the lepton $e^\pm$
to the $Z$ boson.
The degree of longitudinal polarisation of the incoming lepton is
denoted as $P_e$.
%, and the $P_e$ related terms are only present for non-zero
%values, i.e.\ when the lepton beam is longitudinally polarised, as in HERA-II.
The $\Qsq$-dependent coefficient
$\varkappa_Z$ accounts for the $Z$-boson propagator,
\begin{equation}
  \varkappa_Z(\Qsq)
  = \frac{\Qsq}{\Qsq+m^2_Z}
  \frac{1}{4\sw \cos^2\theta_W}
  = \frac{\Qsq}{\Qsq+m^2_Z}
  \frac{\gf m_Z^2}{2\sqrt{2}\pi\alpha}~.
\end{equation}
It can be normalised using the
weak mixing angle, $\sw=1-\mW^2 / \mZ^2$, i.e.\ using the $W$ and $Z$ boson
masses, \mW\ and \mZ, or the
Fermi coupling constant $\gf$, which is measured with high precision
in muon-decay experiments~\cite{Tishchenko:2012ie}.
The structure functions are related to linear combinations of the
quark and anti-quark momentum distributions, $xq$ and $x\bar{q}$.
For instance the $F_2$ and $xF_3$ structure functions in
the naive quark-parton model, i.e.\ at LO in QCD, are:
\begin{align}
  \left[F_2,F_2^{\gamma Z},F_2^Z\right]
  &= x\sum_q\left[Q_q^2,2Q_q\vq,\vq\vq+\aq\aq \right]\{q+\bar{q}\}~,
  \label{eq:last1}
  \\
  x\left[F_3^{\gamma Z},F_3^Z\right]
  &= x\sum_q\left[2Q_q\aq,2\vq\aq\right]\{q-\bar{q}\}~.
  \label{eq:last2}
\end{align}
The axial-vector and vector couplings of the quarks $q$ to the $Z$ boson,
$g^q_A$ and $g^q_V$, depend on the electric charge, $Q_q$, in units of the
positron charge,
and on the third component of the weak-isospin of the quarks, \Itq.
In terms of $\sw$, they are given by the standard EW theory:
\begin{align}
  g_A^q &= \Itq
  \label{eq:gA-LO} \,, \\
  g_V^q &= \Itq - 2 Q_q \sw
  \label{eq:gV-LO} \,.
\end{align}
The same formulae also apply to the lepton couplings $g^e_{A/V}$.

Universal higher-order corrections, to be discussed below,
can be taken into account by introducing $\Qsq$-dependent form
factors $\rho_{\text{NC}, q}$ and $\kappa_{\text{NC}, q}$ \cite{Olive:2016xmw}, replacing
equations~\eqref{eq:gA-LO} and \eqref{eq:gV-LO}) by
\begin{align}
  g_A^q &= \sqrt{\rho_{\text{NC}, q}} \Itq
  \label{eq:gA-NLO} \, ,
  \\
  g_V^q &= \sqrt{\rho_{\text{NC}, q}} \left( \Itq - 2 Q_q \kappa_{\text{NC}, q}\sw \right)
  \label{eq:gV-NLO} \,.
\end{align}

The CC cross section at LO is written as
\begin{equation}
  \frac{d^2\sigma^{\rm CC}(e^\pm p)}{dxd\Qsq}
  = \left(1 \pm P_e\right)
  \frac{\gf^2}{4\pi x}
  \left[\frac{m_W^2}{m_W^2+\Qsq}\right]^2
  \left(Y_+ W_2^\pm(x,\Qsq) \mp Y_{-} xW_3^\pm(x,\Qsq)
  - y^2 W_{\rm L}^\pm(x,\Qsq)\right)~.
  \label{eq:cc-cs}
\end{equation}
In the quark-parton model, $W_{\rm L}^\pm = 0$, and the structure
functions $W_2^\pm$ and $xW_3^\pm$ are obtained from the parton
distribution functions. For electron scattering, only
positively charged quarks contribute:
\begin{equation}
  W_2^- =
  x \left( U + \overline{D} \right)
  \, ,
  \quad xW_3^- =
  x \left( U - \overline{D} \right)
  \, ,
  \label{eq:w23el-LO}
\end{equation}
while negatively charged quarks contribute to positron scattering:
\begin{equation}
  W_2^+ =
  x \left( \overline{U} + D \right)
  \, ,
  \quad xW_3^+ =
  x \left( D - \overline{U} \right)
  \, .
  \label{eq:w23po-LO}
\end{equation}
Below the top-quark threshold, one has
\begin{equation}
  U = u+c\, , \quad
  \overline{U} = \bar{u} + \bar{c}\, , \quad
  D = d+s\, , \quad
  \overline{D} = \bar{d} + \bar{s} \, .
\end{equation}

Higher-order EW corrections are collected in form factors
$\rho_{\text{CC}, eq/e\bar{q}}$. They modify the LO expressions
equations~\eqref{eq:w23el-LO} and \eqref{eq:w23po-LO}
as
\begin{align}
  W_2^- &=
  x \left( \rho^2_{\text{CC}, eq} U + \rho^2_{\text{CC},e\bar{q}} \overline{D} \right)
  \, ,
  \quad xW_3^- =
  x \left( \rho^2_{\text{CC},eq} U - \rho^2_{\text{CC},e\bar{q}} \overline{D} \right)
  \, ,
  \label{eq:w23el-NLO}
\\
  W_2^+ &=
  x \left( \rho^2_{\text{CC},eq} \overline{U}+ \rho^2_{\text{CC},e\bar{q}} D \right)
  \, ,
  \quad xW_3^+ =
  x \left( \rho^2_{\text{CC},e\bar{q}} D - \rho^2_{\text{CC},eq} \overline{U} \right)
  \, .
  \label{eq:w23po-NLO}
\end{align}


In the on-shell (OS) scheme~\cite{Sirlin:1980nh,Sirlin:1983ys},
the independent parameters of the SM EW theory are
determined by the fine structure constant $\alpha$ and the masses
of the gauge bosons, the Higgs boson $m_H$, and
the fermions $m_f$. The weak mixing angle is then fixed, and $\gf$ is a prediction, given by
\begin{equation}
  \gf=\frac{\pi\alpha}{\sqrt{2}m_W^2}
  %  \left(1-\frac{m_W^2}{m_Z^2}\right)^{-1}
  \frac{1}{\sw}
  \frac{1}{(1-\dr)} \, ,
  \label{eq:gf-mw-mz}
\end{equation}
where higher-order corrections enter through the
quantity $\dr = \dr(\alpha, m_W, m_Z, m_H, m_t, \ldots)$
\cite{Sirlin:1980nh}, which describes corrections to the muon
decay beyond the tree-level~\cite{Bohm:1986rj,Hollik:1988ii}.

The $\rho_\text{NC}$, $\kappa_\text{NC}$ and $\rho_\text{CC}$ parameters are introduced to cover
the universal higher-order EW corrections described by
loop insertions in the boson propagators. The $\rho_\text{NC}$ parameters absorb
$Z$-boson propagator corrections combined with higher-order
corrections entering the $\gf$-$\mW$-$\sw$ relation,
equation~\eqref{eq:gf-mw-mz}, while the $\kappa_\text{NC}$ parameters absorb
one-loop $\gamma Z$ mixing propagator corrections. In addition,
there are higher-order corrections to the photon propagator
which can be taken into account by using the running fine structure
constant. Non-universal corrections due to vertex one-loop
Feynman graphs and box diagrams are added separately to the
NC cross sections. For the CC cross sections, both universal
and non-universal corrections can be combined into the form factors
$\rho_{\text{CC},eq/e\bar{q}}$.
The dominating corrections in this case are due
to loop insertions in the $W$-boson propagator.


One-loop EW corrections have been calculated
in refs.~\cite{Bohm:1986na,Bardin:1988by,Hollik:1992bz}
for NC and in refs.~\cite{Bohm:1987cg,Bardin:1989vz} for CC
scattering (see also ref.~\cite{Heinemann:1998kk} for a study of
numerical results).
The present analysis uses the implementation
of EW higher-order corrections in the program EPRC described in ref.~\cite{Spiesberger:1995pr}.
%DH: For us the loops in the propagators are relevant and they give raise to four charge form factors owing to the four exchanges WW, ZZ, $\gamma\gamma$ and $\gamma$Z.\
 Here the parameters needed to make the gauge theory predictive can be introduced and related to observables.
The size of the purely weak one-loop
corrections for the
differential cross sections is displayed in
figure~\ref{fig:EWcorrections} for
selected values of \Qsq\ for $e^+p$ scattering.
It includes the $\rho_\text{NC/CC}$ and
$\kappa_\text{NC}$ form factors, as well as contributions from vertex
and box graphs.
The correction factors for electron
scattering and their dependence on the electron polarisation
differ by less than a per cent and do not change the overall
picture. Higher-order QED corrections due to real and virtual
emission of photons, as well as vacuum polarisation, i.e.\ the
running of the fine structure constant, also have to be taken
into account \cite{Kwiatkowski:1990es,Charchula:1994kf}.
These effects, however, had been considered for the cross section measurement  and are
therefore not included here.

%-----------------------------------------------------------------------
\begin{figure}[tb]
  \begin{center}
%    \includegraphics[width=0.495\textwidth]{{plots/EWHO_NC.ep.P0}.pdf}\hskip0.01\textwidth
%    \includegraphics[width=0.495\textwidth]{{plots/EWHO_CC.ep.P0}.pdf}\hskip0.01\textwidth
  \end{center}
  \caption{
    Size of the purely weak one-loop corrections for the $e^+p$
    unpolarised inclusive NC DIS (left) and CC DIS (right) cross sections
    at selected values of \Qsq\ as a function of \xbj.
    QED corrections due to real and virtual photons and corrections from
    the vacuum polarisation (the running of $\alpha$) are not included.
    %since those corrections have been removed in the previous data analysis.
    The corresponding corrections for electron scattering,
    and also their dependence on the lepton beam polarisation, are
    overall very similar and differ by less than a per cent.
    %    The dominant contributions for CC cross sections arise through the calculation of the Fermi Constant, \gf, where 1-loop corrections are considered through \dr\
.
  }
  \label{fig:EWcorrections}
\end{figure}
%-----------------------------------------------------------------------
In the on-shell scheme, used in this analysis,
the higher-order correction factors $\rho_\text{NC}$, $\kappa_\text{NC}$ and
$\rho_\text{CC}$ are calculated as a function of $\alpha$ and the
input mass values.
They depend quadratically on the top-quark
mass through $\Delta\rho_t\sim m_t^2$, and logarithmically on the
Higgs-boson mass, $\Delta\rho_H\sim\ln\left(m^2_H/\mW^2\right)$.
On the $Z$ pole they amount to about 4\%.
For DIS at HERA they are of similar size, but they exhibit a
non-negligible \Qsq-dependence~\cite{Spiesberger:1993jg}.
%Alternatively, in the modified on-shell scheme \cite{Marciano:1980pb}, $\gf$ together with $m_Z$ are taken as independent parameters of the SM EW theory. The $W$-boso\
n mass $m_W$ and the weak mixing angle \sw\ are determined from equation~\eqref{eq:gf-mw-mz}. The $\gf$ sets the normalization of the CC cross section as well as the $\
Z$-boson exchange contributions of the NC cross section. This prescription exploits the precisely measured muon lifetime, thus allowing to obtain predictions with smal\
ler uncertainties due to errors from input parameters. In addition,
In  a modified version of the on-shell scheme~\cite{Marciano:1980pb},
commonly used in QCD analyses of DIS data, the Fermi constant can be
used to fix the input parameters replacing the $W$-boson mass as an
input parameter.
In that case the one-loop corrections are very small,
i.e.\ $\rho_{\text{CC},eq/e\bar{q}}$ deviate from 1 by a few per mille.

%The different prescriptions to fix the free parameters of the
%electroweak theory in the on-shell
%or in the modified on-shell
%scheme lead to different sensitivities of the measured cross sections
%to the EW parameters.
%Both schemes will be used in this analysis to perform tests of the SM.


Many extensions of the SM predict modifications of the weak
neutral-current couplings. They can be described conveniently
by introducing additional parameters $\rhop{}$ and $\kapp{}$, thus
modifying the SM corrections.
Also for charged current cross sections, similar $\rhopW{}$ parameters
describing non-standard modifications of the CC couplings can be
introduced.
The $\rhop{}$,  $\kapp{}$ and $\rhopW{}$ are introduced
through the following replacements in
equations~\eqref{eq:gA-NLO}, \eqref{eq:gV-NLO}, \eqref{eq:w23el-NLO}
and \eqref{eq:w23po-NLO}:
\begin{align}
  \rhop{}    &\rightarrow \rhop{}\rho_\text{NC}  \label{eq:rhozkapz1}~,\\
  \kapp{}    &\rightarrow \kapp{}\kappa_\text{NC}  \label{eq:rhozkapz2}~,\\
  \rhopW{} &\rightarrow \rhopW{}\rho_\text{CC} \label{eq:rhopW}~.
\end{align}
In the SM, the parameters $\rhop{}$, $\kapp{}$ and $\rhopW{}$ are defined to be
1.
Various models with physics beyond the SM predict typical
flavour-dependent deviations from 1 and therefore distinct parameters
for quarks ($\rhop{,q}$ and $\kapp{,q}$) and for leptons ($\rhop{,e}$ and $\kapp{,e}$) are considered.
These parameters may also depend on the energy scale.
Precision EW measurements on the
$Z$ resonance are sensitive to the NC couplings
at $m_Z$~\cite{ALEPH:2005ab}, while DIS is also probing their \Qsq\ dependence.
For CC there could be independent modifications (\rhopW{}) for
the lepton and quark couplings for each generation.
However, only the product of lepton times quark couplings appears in the final
expression for the cross section and therefore the same non-standard
coupling for all generations is assumed here.
Nonetheless, new 4-fermion operators can introduce
a difference between electron-quark and electron-antiquark scattering,
and thus two distinct parameters \rhopW{,eq}\ and \rhopW{,e\bar{q}}
are considered.
These possibly scale-dependent parameters allow for additional
tests of the SM couplings.

%-----------------------------------------------------------------------
%   Data
%-----------------------------------------------------------------------
%\clearpage
\section{H1 inclusive DIS cross section data}
\label{sec:data}
This study is based on the entire set of measurements of inclusive NC
and CC DIS cross sections by the H1 Collaboration, using data samples
for $e^+p$ and $e^-p$ taken in HERA-I and HERA-II.
The measurements are subdivided into two kinematic ranges,
corresponding to different subdetectors where the leptons with small
and large scattering angles are identified: low- and medium-\Qsq\ for
values of \Qsq\ typically
smaller than $150\,\GeVsq$ and high-\Qsq for larger values up to 50\,000\,\GeVsq.
A summary of the data sets used is given in table~\ref{tab:table1}.
\begin{table}[tb]
  \footnotesize
  \centering
  \begin{tabular}{rl|cccccc}
   \hline
   \multicolumn{2}{c|}{Data set} & \Qsq-range & $\sqrt{s}$ & \Lumi & No. of & Polarisation & Ref.  \\
                             &  & [{\GeVsq}]   & [{\GeV}] & [{${\rm pb}^{-1}$}] &     data points   & [\%]  &   \\
   \hline
    1 & \ep\ combined low-\Qsq  & (0.5) 8.5 -- 150 & 301,319 & 20,\,22,\,97.6  &  94 (262) & --  &  \cite{Collaboration:2010ry} \\
    2 & \ep\ combined low-$E_p$ & (1.5) 8.5 -- 90  & 225,252 & 12.2,\,5.9 & 132 (136) & --  &  \cite{Collaboration:2010ry} \\

    3 & \ep\ NC 94--97  & 150 -- 30\,000 & 301 & 35.6 & 130 &  --& \cite{Adloff:1999ah} \\
    4 & \ep\ CC 94--97  & 300 -- 15\,000 & 301 & 35.6 &  25 &  -- & \cite{Adloff:1999ah} \\
   \hline
   \end{tabular}
  \caption{Data sets used in the combined EW and QCD fits. For
    (CC) cross sections.}
  \label{tab:table1}
\end{table}


%-----------------------------------------------------------------------
%        Methodology
%-----------------------------------------------------------------------
%\clearpage
\section{Methodology}
The EW parameters are determined in fits of the predictions
to data, where in addition to the EW parameters of interest also
parameters of the PDFs  are determined in order to account for PDF
uncertainties.
The fits are denoted according to their fit
parameters, for instance `\mW+PDF' denotes a determination of
\mW\ together with the parameters of the PDFs.

A dedicated determination of the PDFs in this analysis is important,
since all state-of-the-art PDF sets were determined using H1 data, while
assuming that the EW parameters take their SM values.
Hence, the use of such PDF sets could bias the results.
Furthermore, PDF sets which include the H1 data suffer from the
additional complication that the same data were to be used twice, thus
leading to underestimated uncertainties.

The parameterisation of the PDFs %, the procedure
follows closely the approach of
ref.~\cite{Abramowicz:2015mha}, where the PDF set
HERAPDF2.0\,\footnote{
HERAPDF2.0 is determined from combined inclusive NC and CC data from
the H1 and ZEUS experiments assuming unpolarised lepton beams.
}
was obtained, using EW parameters determined from other experiments.
The parameterisation uses five functional forms with altogether 13 fit
parameters, defined at the starting scale $Q_0^2=1.9\,\GeVsq$.
The scale dependence of PDFs is evaluated using the DGLAP formalism.
%The parameterisation and calculations of the PDFs follows closely the methodology as outlined in ref.~\cite{Abramowicz:2015mha}.

As opposed to the HERAPDF2.0 analysis,
the {\sc Alpos} fitting framework~\cite{Andreev:2017vxu} is used in
the present analysis.
The cross section predictions have been validated against the xFitter
framework~\cite{Aaron:2012qi,Alekhin:2014irh,Abramowicz:2015mha},
which is the successor of the H1Fitter framework~\cite{Aaron:2009kv}.
The structure functions are obtained in the zero-mass
variable-flavour-number-scheme at NNLO in QCD using the program
QCDNUM~\cite{Botje:2010ay,Botje:2016wbq}.
The one-loop EW corrections are included in an updated version of the
program EPRC~\cite{Spiesberger:1995pr}, while the data have already
been corrected for higher-order QED radiative
effects, as outlined in section~\ref{sec:data}.
%Higher-order QED corrections have been considered already for the cross section measurements.


The goodness of fit,
%quantity,
\chisq, is derived from a likelihood function assuming the quantitites
%probability densities of the relative uncertainties
to be normal distributed in terms of relative uncertainties~\cite{Andreev:2014wwa,Andreev:2017vxu},
which is equivalent to log-normal distributed quantities in terms of
absolute uncertainties.
The log-normal distribution is strictly positive and a good
approximation of a Poisson distribution.
The latter is important, since in the kinematic domain where the data
exhibit the highest sensitivity to the EW parameters, the statistical
uncertainties may become sizeable and dominating.
The \chisq\ is calculated as
\begin{equation}
  \chi^2 = \sum_{ij}\log\tfrac{\varsigma_i}{\tilde\sigma_i} V_{ij}^{-1}\log\tfrac{\varsigma_j}{\tilde\sigma_j}\,,
\end{equation}
where the sum runs over all data points with measured cross sections
$\varsigma_{i}$ and the corresponding theory predictions,
$\tilde\sigma_i$. The covariance matrix $V_{ij}$ is constructed from all
relative uncertainties, taking also correlated uncertainties between
the data sets into account~\cite{Aaron:2012qi}.
The beam polarisation measurements provide four additional
data points, included in the vector $\varsigma$, with their
uncertainties~\cite{Sobloher:2012rc} and four corresponding
parameters in the fit.




%-----------------------------------------------------------------------
%                   Results
%-----------------------------------------------------------------------
%\clearpage
\section{Results}
This section reports the results of different fits, starting with mass
determinations in section~\ref{sec:mass}, followed by weak NC coupling
determinations in section~\ref{sec:couplings} and finally the study of
$\rhop{}$, $\kapp{}$ and $\rhopW{}$ parameters in section~\ref{sec:ff}.
%and of their possible scale dependence in section~\ref{sec:scale}.

\subsection{Mass determinations}\label{sec:mass}
%
% DH: In the onshell scheme with the independent input parameters ($\alpha$,\mZ,\mt,\mH) the HERA data constrain the mass of the W to …  The analysis includes a detailed investigation of the weak propagators. With a different choice of independent input parameters also the masses of the Higgs and the top quark can be constrained, which is less accurate as a consequence of the intrinsic logarithmic dependence of the radiative corrections. Fitting simultaneously two parameters leads to larger limits, but restricts strongly the available phase space due to their strong intrinsic correlation.}
%
The masses of the $W$ and $Z$ bosons, as well as the top-quark mass
are determined using different prescriptions to fix the fit parameters of the EW theory in the on-shell
scheme. The different prescriptions lead to different sensitivities of the measured cross sections
to the EW parameters~\cite{Blumlein:1987fd}.
%or in the modified on-shell
%scheme lead to different sensitivities of the measured cross sections
%to the EW parameters.
%Both schemes will be used in this analysis to perform tests of the SM.
The results are summarised in table~\ref{tab:masses}.
\begin{table}[hbt]
  \footnotesize
  \centering
   \begin{tabular}{lc@{$\,=\,$}lc}
   \hline
   Fit parameters & \multicolumn{2}{c}{Result} & Independent input parameters \\
   \hline %  ---------------------------------
   %%% mW-mZ
   \mW+PDF  & $\mW$ & $80.508\pm0.070_{\rm stat}\pm 0.055_{\rm syst}\pm0.073_{\PDF}\,\GeV$ & $\alpha$, $m_Z$, $m_t$, $m_H$, $m_f$ \\ %final value (22.3.18)
   \mWprop+PDF & $\mWprop $ & $ 80.44\pm0.67_{\rm stat}\pm0.17_{\rm syst}\pm0.38_{\PDF}\,\GeV$ & $\alpha$, $m_W$, $m_Z$, $m_t$, $m_H$, $m_f$ \\
   \mWGfW+PDF &  $\mWGfW $ & $ 82.02\pm0.51_{\rm stat}\pm0.44_{\rm syst}\pm0.38_{\PDF}\,\GeV$ & $\alpha$, \gf, $m_t$ $m_H$, $m_f$ \\
   \mZ+PDF & $\mZ $ & $ 91.099\pm0.064_{\rm stat}\pm 0.050_{\rm syst}\pm0.070_{\PDF},\GeV$ & $\alpha$, $m_W$, $m_t$, $m_H$, $m_f$ \\ %final value (22.3.18)
   \mt+PDF & $\mt $ & $ 157\pm10_{\rm stat}\pm12_{\rm syst}\pm15_{\rm PDF}\pm15_{\mW}\,\GeV$ & $\alpha$, $m_W$, $m_Z$, $m_H$, $m_f$ \\ %final value (22.3.18)
   \hline %  ---------------------------------
   \end{tabular}
   \caption{Results for five combined fits of mass parameters together with PDFs.
   The multiple uncertainties correspond to statistical (stat), experimental
   systematic (syst) and PDF uncertainties. The last uncertainty
   contribution to the $m_t$ determination is due to the uncertainty of
   the $m_W$ mass.}
   \label{tab:masses}
\end{table}

In the combined \mW+PDF fit,
where
%the on-shell scheme is employed and thus
$\alpha$, \mZ, \mt, \mH\ and $m_f$ are taken as external input values~\cite{Olive:2016xmw}, the EW parameter \mW\ is determined to be
\begin{equation}
  \mW = 80.508\pm0.070_{\rm stat}\pm0.055_{\rm syst}\pm0.074_{\PDF}=80.508\pm 0.115_{\rm tot}\,\GeV~.
\end{equation}
%The decomposition of the uncertainties is performed by
%\begin{compactitem}
%\begin{itemize}
%  \item
and the expected uncertainty\,\footnote{The expected
  uncertainty is obtained from a re-fit using the Asimov data
  set and the data uncertainties~\cite{Cowan:2010js}.}.
The total (tot) uncertainty is improved by
about a factor of two in comparison to the earlier result based on
HERA-I data only~\cite{Aktas:2005iv}.
The uncertainty decomposition is derived by switching off the
uncertainty sources subsequently or repeating the fit with fixed PDF parameters\,\footnote{
    The PDF uncertainty contains both a statistical and a
    systematic component, but the systematic component
    dominates.}
is 0.118\,\GeV.
Other uncertainties due to uncertainties of the input masses
(\mZ, \mt, \mH) and theoretical uncertainties, e.g.\ from
incompletely known higher-order terms in \dr,
or model and parameterisation uncertainties of the
PDF fit, are all found to be negligible with respect to the
experimental uncertainty.
The correlation of \mW\ with any of the PDF parameters is always weak,
with absolute values of the correlation coefficients below 0.2.
The global correlation coefficient~\cite{James:1975dr} of \mW\
in the EW+PDF analysis is 0.64.
The \mW\ sensitivity arises predominantly from the CC data, with the most
important constraint being the normalisation, \gf\ (see
equations~\eqref{eq:cc-cs} and \eqref{eq:gf-mw-mz}).
The highest sensitivity of the H1 data to \mW\ is at a \Qsq\ value of
about 3800\,\GeVsq.

%
%The result on \mW\ is compared in figure~\ref{fig:mW} to measurements by other single experiments~\cite{Barate:1997bf,Acciarri:1997ra,Ackerstaff:1996nk,Abdallah:2008ad,Abazov:2012bv,Aaltonen:2012bp,Aaltonen:2013vwa,Schael:2013ita,Aaboud:2017svj} and to a previous world average value~\cite{Aaltonen:2013iut,Olive:2016xmw}.
%The value of \mW\ is found to be consistent with other
%determinations.
%although with larger uncertainties than that of the other determinations displayed.
%
The result for \mW\ is compared to determinations from other single
experiments~\cite{Barate:1997bf,Acciarri:1997ra,Ackerstaff:1996nk,Abdallah:2008ad,Abazov:2012bv,Aaltonen:2012bp,Aaltonen:2013vwa,Aaboud:2017svj}
in figure~\ref{fig:mW}, and is found to be consistent with these
% Schael:2013ita,
as well as with the world average value
of $80.385\pm0.015\,\GeV$~\cite{Aaltonen:2013iut,Olive:2016xmw}.
%
The $W$-mass determination in the space-like regime at HERA can be interpreted
as an indirect constraint on
%\mW, or equivalently through equation~\eqref{eq:gf-mw-mz} as a measurement of
\gf\ through equation~\eqref{eq:gf-mw-mz}, however in a process at large momentum
transfer~\cite{Blumlein:1987fd,Beyer:1995pw,CooperSarkar:1998ug}.
%
Using the world average value of
%$\mZ=91.1876\pm0.002\,\GeV$
\mZ~\cite{ALEPH:2005ab,Olive:2016xmw},
the result obtained here, $\mW=80.508\pm 0.115\,\GeV$,
represents an indirect determination of the weak mixing angle in the
OS scheme as
$\sw=0.22052\pm0.00223$. % value as of 22.3.18 (80.5077 +/- 0.115238
                         % -> 0.220523 +/- 0.00223147)
%  Gf would be: 1.17515e-05  +/- 7.76105e-08
%               1.175(8)     x 10-5 GeV-2
%  muLan has:   1.1663787(6) x 10-5 GeV-2

%-----------------------------------------------------------------------
\begin{figure}[tb]
  \begin{center}
%    \includegraphics[width=0.48\textwidth]{{plots/Wmass_summary}.pdf}\hskip0.01\textwidth
  \end{center}
  \caption{
    Value of the $W$-boson mass compared to results obtained by the ATLAS, ALEPH, CDF,
    D0, DELPHI, L3 and OPAL experiments, and the world average value.
    The inner error bars indicate statistical uncertainties and the
    outer error bars full uncertainties.
}
\label{fig:mW}
\end{figure}



%-----------------------------------------------------------------------
%                            Summary
%-----------------------------------------------------------------------
\section{Summary}
\label{sect:Conclusion}



%%%%%%%%%%%%%%%%%%%%%%%%%%%%%%%%%%%%%%%%%%%%%%%%%%%%%%%%%%%%%%%%%%%%%

\section*{Acknowledgements}
Acknowledgements: Z.~Zhang, A.~Sch\"oning, M.~Klein, S.~Schmitt


%%%%%%%%%%%%%%%%%%%%%%%%%%%%%%%%%%%%%%%%%%%%%%%%%%%%%%%%%%%%%%%

%%%%%%%%%%%%%%%%%%%%%%%%%%%%%%%%%%%%%%%%%%%%%%%%%%%%%%%%%%%%%%%%%



%%======================= References ==========================%%
%%%%%%%%%%%%%%%%%%%%%%%%%%%%%%%%%%%%%%%%%%%%%%%%%%%%%%%%%%%%%%%%%

\clearpage

\begin{flushleft}
\bibliography{lhec_ew}
\end{flushleft}


\end{document}

%%======================= References ==========================%%
%%%%%%%%%%%%%%%%%%%%%%%%%%%%%%%%%%%%%%%%%%%%%%%%%%%%%%%%%%%%%%%%%
\clearpage
\begin{flushleft}
\bibliography{lhec_ew}
\end{flushleft}


\end{document}
